% Simulación basada en agentes

\chapter*{Simulación basada en Agentes} \label{cap1}
\addcontentsline{toc}{chapter}{Simulación basada en Agentes}

\pagenumbering{arabic}

\begin{flushright}
\begin{minipage}{7.85cm}
    {\em Creo que para final de siglo el uso de las palabras y la opinión
    general habrán cambiado tanto, que uno podrá hablar sobre máquinas pensantes
    sin que le contradigan.} \\ Alan M. Turing
\end{minipage}
\end{flushright}

\vspace*{5mm}

\section*{Agentes Inteligentes}

% http://en.wikipedia.org/wiki/Intelligent_agent#A_variety_of_definitions

Una posible definición muy resumida de agente inteligente sería la siguiente:
{\em Un agente inteligente es una entidad autónoma que observa y actúa sobre un
entorno, y dirige su actividad en pos de uno o más objetivos}. Por supuesto
esta definición no abarca muchos matices, pero nos da una idea general que nos
sirve como punto de partida. La variedad de agentes es muy amplia, así como
su complejidad, que puede ir desde un sencillo agente puramente reactivo a uno
complejo que imite a una persona, por ejemplo.

Existen muchas definiciones de agentes inteligentes, todas ellas diferentes,
pero con puntos en común. Los puntos que se pueden extraer como fundamentales
para considerar a un proceso como agente inteligente son:

\begin{description}
 \item[Autonomía]Deben ser capaces de funcionar sin la intervención del usuario.
 \item[Reactividad]Han de poder observar su entorno, y responder a los eventos
 o cambios que se produzcan.
 \item[Pro-actividad]Para conseguir sus objetivos han de tomar decisiones y
 realizar acciones según una estrategia.
 \item[Comunicación]Deben poder comunicarse entre sí, compartir información, e
 incluso, cooperar para la consecución de sus objetivos.
\end{description}

Aunque existen agentes puramente reactivos, es difícil aplicarles la etiqueta
de {\em inteligentes}. Un grado de pro-actividad es imprescindible, aunque se
hace necesario encontrar un equilibrio entre la reactividad y la búsqueda
activa de objetivos.

\subsection*{Tipos de Agentes}

% http://en.wikipedia.org/wiki/Intelligent_agent#Classes_of_intelligent_agents

\subsubsection*{Reactivos}

\subsubsection*{Reactivos basados en Modelo}

\subsubsection*{Basados en Objetivos}

\subsubsection*{Basados en Utilidad}

\subsubsection*{Con capacidad de Aprendizaje}

\section*{Sistemas Multiagente (SMA)}

% http://en.wikipedia.org/wiki/Multi-agent_system

\subsection*{Métodos de comunicación}
