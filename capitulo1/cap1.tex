% Simulación basada en agentes

\chapter*{Simulación basada en Agentes} \label{cap1}
\addcontentsline{toc}{chapter}{Simulación basada en Agentes}

\pagenumbering{arabic}

\begin{flushright}
\begin{minipage}{7.85cm}
    {\em Creo que para final de siglo el uso de las palabras y la opinión
    general habrán cambiado tanto, que uno podrá hablar sobre máquinas pensantes
    sin que le contradigan.} \\ Alan M. Turing
\end{minipage}
\end{flushright}

\vspace*{5mm}

\section*{Agentes Inteligentes}

% http://en.wikipedia.org/wiki/Intelligent_agent#A_variety_of_definitions

Una posible definición muy resumida de agente inteligente sería la siguiente:
{\em Un agente inteligente es una entidad autónoma que observa y actúa sobre un
entorno, y dirige su actividad en pos de uno o más objetivos}. Por supuesto
esta definición no abarca muchos matices, pero nos da una idea general que nos
sirve como punto de partida. La variedad de agentes es muy amplia, así como
su complejidad, que puede ir desde un sencillo agente puramente reactivo a uno
complejo que imite a una persona, por ejemplo.

Existen muchas definiciones de agentes inteligentes, todas ellas diferentes,
pero con puntos en común. Los puntos que se pueden extraer como fundamentales
para considerar a un proceso como agente inteligente son:

\begin{description}
 \item[Autonomía]Deben ser capaces de funcionar sin la intervención del usuario.
 \item[Reactividad]Han de poder observar su entorno, y responder a los eventos
 o cambios que se produzcan.
 \item[Pro-actividad]Para conseguir sus objetivos han de tomar decisiones y
 realizar acciones según una estrategia.
 \item[Comunicación]Deben poder comunicarse entre sí, compartir información, e
 incluso, cooperar para la consecución de sus objetivos.
\end{description}

Aunque existen agentes puramente reactivos, es difícil aplicarles la etiqueta
de {\em inteligentes}. Un grado de pro-actividad es imprescindible, aunque se
hace necesario encontrar un equilibrio entre la reactividad y la búsqueda
activa de objetivos.

\subsection*{Tipos de Agentes}

% http://en.wikipedia.org/wiki/Intelligent_agent#Classes_of_intelligent_agents

Existen varios tipos de agentes según sus características y capacidades. Lo que
sigue es sólo una de las posibles clasificaciones.

\subsubsection*{Reactivos}

Los agentes puramente reactivos actúan únicamente en base a su percepción. La
funcionalidad del agente está basada en {\em la regla de condición-acción}: si
tal condición entonces tal acción. Este tipo de agentes es particularmente
interesante cuando el entorno es observable en su totalidad.

También es posible que un agente reactivo contenga información de su estado en
cada momento, lo que les permite obviar {\em condiciones} que ya hayan activado
{\em acciones} en momentos anteriores.

\subsubsection*{Reactivos basados en Modelo}

Este tipo de agentes es capaz de desenvolverse en entornos que son sólo
parcialmente observables. Almacenan su estado en cada momento manteniendo algún
tipo de estructura que describa el entorno que no se puede percibir.
Obviamente, esta cualidad requiere información sobre como se comporta y
funciona el entorno. Es esta información adicional la que completa las
percepciones del agente.

Un agente reactivo basado en modelo realiza un seguimiento del estado del
entorno en cada momento, usando su modelo interno. Con ambas informaciones (su
modelo interno y sus percepciones) toma las decisiones de la misma manera que un
agente puramente reactivo.

\subsubsection*{Basados en Objetivos}

Los agentes basados en objetivos son agentes basados en modelo que además
almacenan información relativa a situación que son deseables. Gracias a esto el
agente es capaz de escoger de entre múltiples posibilidades aquellas que
cumplen un objetivo, o que al menos ayudan a alcanzar uno.

\subsubsection*{Basados en Utilidad}

\subsubsection*{Con capacidad de Aprendizaje}

\section*{Sistemas Multiagente (SMA)}

% http://en.wikipedia.org/wiki/Multi-agent_system

\subsection*{Métodos de comunicación}
