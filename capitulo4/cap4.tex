% Tecnología elegida (JADE)

\chapter*{Tecnología elegida} \label{cap4}
\addcontentsline{toc}{chapter}{Tecnología elegida}

\pagenumbering{arabic}

\begin{flushright}
\begin{minipage}{7.85cm}
    {\em Cualquier tecnología lo suficientemente avanzada es indistinguible de
    la magia.} \\ Arthur C. Clarke
\end{minipage}
\end{flushright}

\vspace*{5mm}

\section*{Licencias}

% TODO gráfico de compatibilidad de licencias con la GPL v3

\section*{JADE}

\section*{Java}

\section*{JAK}

JAK\footnote{Java API for KML: \url{http://labs.micromata.de/display/jak/Home}}
son un conjunto de librerías para el manejo de ficheros KML desde Java. JAK ha
sido desarrollado por Micromata GmbH (\url{http://www.micromata.de/}).

% TODO Imagen con el logo de JAK

Al igual que en el caso de {\bf JADE} es necesario que JAK sea compatible con
la licencia de nuestro proyecto, la GPL v3. % TODO Link al ap1
Los desarrolladores de JAK optaron por liberar las librerías bajo una licencia
BSD\footnote{Berkeley Software Distribution:
\url{http://www.opensource.org/licenses/bsd-license.php}} de 3 clausulas, o como
también se la conoce, licencia BSD modificada o nueva licencia BSD. La licencia
de JAK se puede consultar en su sitio web
(\url{http://labs.micromata.de/display/jak/Licenses}).

Gracias a JAK podremos generar los ficheros KML con el resultado de la
simulación.

\section*{Otros}

% TODO comentar OSM y el servicio web de USGS

\subsection*{Jcoord}

Para el manejo de coordenadas geográficas utilizaremos
Jcoord\footnote{\url{http://www.jstott.me.uk/jcoord/}}, un paquete de clases que
nos permitirán convertir entre formatos de coordenadas, calcular distancias,
etc. Jcoord ha sido desarrollado por Jonathan Mark Stott, y tal como se puede
consultar en su sitio web está licenciado bajo GPL v2, compatible con nuestra
GPL v3. %TODO link ap1

Además del habitual sistema de coordenadas latitud/longitud
WGS84\footnote{Sistema Geodésico Mundial 1984} soporta UTM\footnote{Sistema de
Coordenadas Universal Transversal de Mercator}.
% TODO las lat/lng normales están en WGS84? creo que sí...

A diferencia de {\bf JADE} o {\bf JAK}, integraremos directamente el código en
el simulador, por lo que no será una dependencia. De esta manera podremos
modificar y adaptar el paquete a nuestras necesidades.

\subsection*{OpenWFE}

\subsection*{Java CSV}