% Tecnología elegida (JADE)

\chapter*{Tecnología elegida} \label{cap4}
\addcontentsline{toc}{chapter}{Tecnología elegida}

\pagenumbering{arabic}

\begin{flushright}
\begin{minipage}{7.85cm}
    {\em Cualquier tecnología lo suficientemente avanzada es indistinguible de
    la magia.} \\ Arthur C. Clarke
\end{minipage}
\end{flushright}

\vspace*{5mm}

\section*{Licencias}

% TODO

\section*{JADE}

\section*{Java}

\section*{JAK}

JAK\footnote{Java API for KML: \url{http://labs.micromata.de/display/jak/Home}}
son un conjunto de librerías para el manejo de ficheros KML desde Java.

% TODO Imagen con el logo de JAK

Al igual que en el caso de {\bf JADE} es necesario que JAK sea compatible con
la licencia de nuestro proyecto, la GPL v3. % TODO Link al ap1
Los desarrolladores de JAK optaron por liberar las librerías bajo una licencia
BSD\footnote{Berkeley Software Distribution:
\url{http://www.opensource.org/licenses/bsd-license.php}} de 3 clausulas, o como
también se la conoce, licencia BSD modificada o nueva licencia BSD. La licencia
de JAK se puede consultar en su sitio
web\footnote{\url{http://labs.micromata.de/display/jak/Licenses}}.

Gracias a JAK podremos generar los ficheros KML con el resultado de la
simulación.

\section*{Otros}

% TODO comentar OSM y el servicio web de USGS

\subsection*{JCoord}

\subsection*{OpenWFE}

\subsection*{Java CSV}