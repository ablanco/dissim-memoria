\chapter{Implementación (II): Ingeniería de Escenarios} \label{cap5b}
%\addcontentsline{toc}{chapter}{Implementación (II): Ingeniería de Escenarios}

\begin{flushright}
\begin{minipage}{7.85cm}
    {\em blablablabla} \\  Alguien % TODO
\end{minipage}
\end{flushright}

\vspace*{5mm}

\section{Introducción}

El escenario de simulación es la manera que tenemos de decirle a nuestro
simulador qué es lo que queremos simular precisamente. Dado que el simulador es
versátil y tiene capacidad de simular, en principio, cualquier lugar, hay que
definir una serie de parámetros de entrada.

En un escenario hemos de especificar cuál es el área con la que se va a
trabajar, cuáles son las entradas de agua y qué agentes humanos intervienen.
Aparte se definen también algunos parámetros técnicos como pueden ser los datos
de conexión a la base de datos de alturas.

Entrando en detalles los elementos de los que consta un escenario son:

\begin{description}
 \item[Nombre y descripción] Los escenarios se nombran para poder
identificarlos con facilidad, este nombre y descripción se muestra también en
los resultados de la simulación.
 \item[Fecha y hora] También se define la fecha y hora del comienzo de la
simulación, lo que permite localizar en el tiempo cada momento de la simulación.
 \item[Resolución temporal] Es necesario definir el tiempo transcurrido entre
dos estados de la simulación, tanto el tiempo simulado (en minutos) que
transcurre, como el tiempo de ejecución (en milisegundos) que se le concede a
la máquina -o máquinas- para simular un paso.
 \item[Área de simulación] El área viene definida por dos coordenadas
geográficas, la coordenada noroeste y la sureste del rectángulo -en realidad
no es un rectángulo si tenemos en cuenta la curvatura terrestre- que definen si
tomamos como diagonal el segmento que las une.
 \item[Resolución espacial] Uno de los parámetros más importantes que hay que
definir es la resolución espacial, tanto superficial como en altura. Hay que
especificar el tamaño de los hexágonos, lo que se hace a través del diámetro de
la circunferencia circunscrita en ellos -los hexágonos son regulares-. También
hay que definir la resolución espacial en altura, es decir, cómo se discretiza
la altura del terreno y del agua; se define como \begin{math}^1/_p\end{math}
metros de precisión donde {\em p} es el parámetro que aparece en el escenario.
 \item[Agentes Entorno] El número de agentes Entorno que se encargaran de
simular el escenario, el número adecuado depende de la cantidad de núcleos o
procesadores del que disponga la máquina donde se lleve a cabo la simulación.
 \item[Base de datos de alturas] En el escenario se incluyen también los datos
necesarios para realizar la conexión a la base de datos donde se almacenan la
elevación del terreno. Estos datos se obtienen de internet, pero se almacenan
localmente para agilizar simulaciones sucesivas en el mismo área.
 \item[Entradas de agua] A la hora de simular una inundación es básico simular
dónde y cuánta agua está entrando al sistema. Dichos datos se encuentran en el
escenario.
 \item[Peatones] La posición y características de los peatones, tales como su
distancia de visión, velocidad y objetivos -refugios que conocen de antemano-.
 \item[Frecuencia de actualización de resultados] Cada cuánto tiempo deben los
agentes Entorno enviar el estado de la simulación a los agentes encargados de
mostrar los resultados, así se define la finura de los resultados obtenidos.
\end{description}

Hay que destacar que los datos definidos en el escenario son los datos
iniciales de la simulación, y que muchos se traducen a agentes, como por
ejemplos las entradas de agua o las personas. Esto permite que en caliente se
puedan añadir, por ejemplo, nuevas fuentes de agua a la simulación, pues se
puede añadir agentes nuevos a una simulación en marcha.

\section{Ejemplo de Escenario}

\lstinputlisting[caption=Escenario de ejemplo]{capitulo5/Ejemplo.scen}

Como se puede ver en el listado el formato de los ficheros escenario es {\em
clave=valor}, y para escribir un array -que son anidables, como se puede ver
también en el ejemplo- {\em clave=[elem,elem,elem...]}.

Lo primero que aparece es el {\em type}, es decir, el tipo de escenario del que
se trata. El valor es el nombre de la clase que hay que instanciar -que debe
heredar de {\em util.Scenario}-, por ahora sólo está implementado el caso de la
inundación, pero el escenario está preparado para múltiples tipos de desastres.

Es importante destacar que el orden en el que aparecen los elementos del
escenario en el fichero es irrelevante, cualquier orden es válido siempre que
aparezcan los elementos mínimos.

A continuación en el ejemplo aparecen el nombre ({\em name}), descripción ({\em
description}), fecha de inicio de la simulación ({\em date}) y hora inicial
({\em hour}).

Los dos siguientes elementos, {\em tick} y {\em realTick}, representan el
tiempo asignado a cada paso de la simulación. El primero se mide en
milisegundos y representa el tiempo de computación, es el que transcurre en
nuestro mundo; y el segundo se mide en minutos y representa cuanto tiempo ha
pasado dentro de la simulación.

{\em NW} y {\em SE} son dos arrays con las coordenadas geográficas de las dos
esquinas del área de simulación. Al igual que el resto de coordenadas que
aparecen en el escenario se escriben con el formato {\em [latitud,longitud]}.
{\em NW} hace referencia a la esquina noroeste, mientras que {\em SE}
referencia a la sureste.

{\em tileSize} corresponde al tamaño de los hexágonos, en metros, mientras que
{\em precision} define la resolución espacial en altura, en
\begin{math}^1/_{precision}\end{math} metros.

Seguido aparece el número de agentes {\bf Entorno}, con la clave {\em numEnvs}.
{\em randomTerrain} es un valor booleano que define si los entornos van a
utilizar datos reales de alturas, o van a generar terrenos aleatorios. Los dos
posibles valores que puede tomar son {\em True} y {\em False}, significando el
primero que el terreno simulado va a ser aleatorio.

A continuación vienen los datos necesarios para la conexión con la base de
datos de alturas. Todos comienzan por {\em DB}. {\em DBServer} es la url, ip,
nombre de fichero, etc, del servidor de base de datos. En este caso se trata de
un servidor que se encuentra en la misma máquina, y al que se accede por red
-de ahí las {\em //}-. {\em DBPort} es el puerto a usar para la conexión. {\em
DBUser} es el usuario con el que identificarse, y {\em DBPass} es la contraseña
en claro a utilizar. {\em DBDriver} especifica qué tipo de servidor se está
utilizando, en este caso se utiliza MySQL\footnote{\url{http://www.mysql.com/}}.
{\em DBDb} es el nombre de la base de datos donde se encuentra la tabla
{\bf Elevations}, dicha tabla está descrita en la sección donde se trata la
obtención de datos de alturas.

No todos los elementos {\em DB} son imprescindibles. Por ejemplo, si se utiliza
el servidor de base de datos de fichero
SQLite\footnote{\url{http://www.sqlite.org/}} no es necesario definir más que
{\em DBServer} y {\em DBDriver}.

Cada línea con la clave {\em waterSource} representa una entrada de agua
diferente al sistema, todas las entradas definidas en el fichero del escenario
se activan nada más comenzar la simulación. Para añadir fuentes de agua que
aparecen un tiempo después del comienzo de la simulación hay que añadir agentes
{\bf Entrada de Agua} en caliente. El array contiene los siguientes datos:
latitud, longitud y cantidad de agua que entra en cada tick medida en unidades
de altura de la simulación. El orden en que aparecen estos elementos es fijo.

Al igual que con las entradas de agua, puede haber múltiples líneas con la clave
{\em person}. Representan a agentes {\bf Peatón} en la simulación. Los
parámetros del array son, en orden, los siguientes: latitud, longitud,
distancia de visión en hexágonos, velocidad en hexágonos, número de clones y un
array de objetivos -con múltiples pares latitud y longitud, aunque en este caso
sólo hay un objetivo-. Todos estos parámetros serán discutidos en la sección de
los agentes {\bf Peatón}.

{\em updateTimeKML} y {\em updateTimeVisor} se miden ambos en milisegundos, y
hacen referencia al tiempo que transcurre entre que los agentes {\bf Entorno}
envían actualizaciones a los generadores de KML y a los visores respectivamente.

\section{Generación de Escenarios}

Aunque es perfectamente posible escribir un fichero escenario con un editor de
textos, resulta más sencillo utilizar el script de lanzamiento de la simulación
para generar uno en un modo interactivo. Dicho modo se explica en detalle en la
sección dedicada al script de lanzamiento.

%%% Local Variables:
%%% mode: latex
%%% TeX-master: "../dissim"
%%% End: