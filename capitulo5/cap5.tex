% Metodología e implementación

\chapter*{Implementación} \label{cap5}
\addcontentsline{toc}{chapter}{Implementación}

\pagenumbering{arabic}

\begin{flushright}
\begin{minipage}{7.85cm}
    {\em Blah blah blah.} \\  Alguien
\end{minipage}
\end{flushright}

\vspace*{5mm}

\section*{Metodología}
%Es un proyecto grande, que hay que organizarse, que usamos una
%metodologia agil, poniendonos metas a corto plazo y repartiendo la carga de
%trabajo
Al ser este un proyecto de tamaño considerable y ser también colaborativo es
necesario establecer una metodología para evitar el trabajo inútil.

Para organizar nuestro trabajo y poder mantener siempre cierto control sobre la
aplicación nos hemos servido de tanto de herramientas, como de buenas prácticas.

Para comenzar y habiendo desarrollado nuestro proyecto en el marco del IV
Concurso Universitario de Software
Libre\footnote{http://www.concursosoftwarelibre.org/} se nos puso a disposición
la una forja
\footnote{https://forja.rediris.es/projects/cusl4-catastrof/} donde pudimos
alojar los comienzos de nuestro proyecto.

Además adoptamos una política de entrevistas frecuentes con los tutores para
así motivarnos y que nos aconsejaran sobre el mejor siguiente paso o en donde
deberíamos profundizar mas o seguir otras vías de investigación.

\subsection*{Metodologías Ágiles}
%Desarrollar pequeñas aplicaciones que sean funcionales y con un limite de
%tiempo, permite tener una version incremental del proyecto y facilita la
%planificación, la priorización de tareas y el reparto.

Una buena planificación no implica el éxito, pero por lo menos te asegura unos
resultados mínimos.

No hemos aplicado ninguna metodología en concreto, si no mas bien un conjunto
de técnicas y buenos métodos.

\subsubsection*{Reuniones de desarrollo}
Lo principal es la comunicación entre los desarrolladores, para ello decidimos
reunirnos y comentar las avances o principales problemas al menos dos veces a
la semana.

Las reuniones no tenían que ser en persona ni muy formales, podían realizarse a
través de mensajería instantánea ayudados con herramientas de pizarras
virtuales. De
esta forma siempre se tiene una idea muy cercana del desarrollo del proyecto sin
perder mucho tiempo.

Además mantiene un fuerte lazo entre los desarrolladores y les permite ayudarse
unos a otros, evitando así estancamientos o frustraciones con problemas
difíciles.
\subsubsection*{Reuniones con los tutores}
Al menos dos veces al mes, dependiendo del ritmo de trabajo y progresos. De
carácter informativo y orientativo.

\subsubsection*{Sistema incremental}
Lo principal era desarrollar una plataforma muy básica y sencilla donde se
pudieran llevar a cabo pruebas, y una vez completara, ir aumentando las
funcionalidades en forma de módulos.

\subsubsection*{Reparto de tareas dinámico} %TODO revisar redacción
Identificar las funcionalidades que se quieren implementar e ir
desarrollándolas en función de su complejidad intentando
siempre el equilibrio en la carga de trabajo.

Hay tareas que son mas sencillas que otras y que requieren mas recursos de
formación e investigación que otras.

Se tiende a la especialización.

\subsection*{Forja RedIRIS}
%La forja esta bien, te permite controlar las tareas abiertas y el tiempo que te
%tiras con ellas, tambien vale para repartirlas
La Forja Red Iris\footnote{https://forja.rediris.es/} dispone de una serie de
herramientas para el alojamiento y la gestión de proyectos de carácter
colaborativo y con vistas de crear una comunidad de desarrolladores.

Gracias a contar con nuestra propia Forja
\footnote{https://forja.rediris.es/projects/cusl4-catastrof/} y con las
herramientas que nos proporciona, entre ellas las mas utilizadas 
\subsubsection*{Subversion}
%Esta bien, pero es muy limitao
sistema de control de versiones
SVN\footnote{http://es.wikipedia.org/wiki/Subversion}, permite llevar un
control sobre los cambios en el código y una copia siempre actualizada de
todos los cambios
\subsubsection*{Planificador de Tareas}
La forja también dispone de una sencilla aplicación de control de tareas.
que entre otras cosas,lo que permite es:
\begin{enumerate}
 \item Crear una tarea, con una breve descripción de la misma.
 \item Asignarlas a un desarrollador, y así poder repartir la carga de trabajo.
 \item Estimación del tiempo necesario, y así poder planificar otras tareas.
 \item Progreso de la tarea, para poder hacer un seguimiento del progreso de la
misma.
 \item Fecha limite de finalización, permite facilitar la planificación
\end{enumerate}
Con estos sencillos parámetros la herramienta incluso genera un diagrama de
GANTT
\footnote{Diagrama de
Gantt:\url{http://es.wikipedia.org/wiki/Diagrama_de_Gantt}} para ayudarte con
la planificación.

\subsection*{Blog}
%Es una buena idea, porque al escribir articulos sirve para pensar mas en lo que
%haces y darle visiblidad, tambien era un requisito para el concurso de software
%libre y sirve para historico y desarrollo  del proyecto.
El llevar al día un blog \footnote{Simulación de Catástrofes
:\url{http://pfc.mensab.com/}} con las actualizaciones y las direcciones
generales sobre el desarrollo proyecto es algo altamente recomendable si se
quiere que el proyecto cree una comunidad, puesto que es una manera muy cómoda y
eficaz de mantener a la comunidad informada de los cambios y el estado del
proyecto.

Además también tiene otras implicaciones mas sutiles, por ejemplo permite
reafirmar las decisiones de diseño o de desarrollo tomadas a lo largo de la
implementación, puesto que queda constancia de las decisiones tomadas y las
razones para llevarlas a cabo.
\subsubsection*{Git}
%Nueva forma de ver las cosas, muchos commits, cambios muy controlaos,
%posibilidad de hacer ramas y mantener siempre una rama estable.

Git es un sistema de control de versiones distribuido que mola mazo %TODO

\section*{Implementación del Escenario de Simulación}
\section*{Implementación de los Agentes en la Simulación}
\subsection*{Agente Creador}
\subsection*{Agente Reloj}
%la necesidad de determinar el tiempo
\subsection*{Agentes Entorno}
\subsection*{Agentes Entrada de Agua}
\subsection*{Agentes Peatón}
\subsection*{Agentes Actualización}
\subsubsection*{Actualización entre entornos adyacentes}
%pasarse el aguita de unos a otros
\section*{Rejilla Hexagonal}
%Discretización del mundo real
\subsection*{Matriz de Altura}
\subsection*{Matriz de Agua}
%Hay que saber cuanta agua tenemos pa moverla
\subsection*{Matriz de Calles}
%Las calles hay que guardarlas en algun lao
\subsection*{Coronas}
%mantenimiento actualizado de la corona
\subsubsection*{Tipos de Calles}
%Especificacion de OSM
\subsection*{Coordenada a Tile}
%conversion de una coordenada a un tile, necesidad de los incrementos
\subsection*{Tile a Coordenada}
%conversion de un tile a una coordenada, tener en cuenta las lineas
%pares/impares
\subsection*{Discretización}
%perdida de datos, caminos no conexos, indeterminacion de las posisciones
\subsection*{Especializacion: Flood Hexagonal Grid}
%cambios de la rejilla por desastre
\section*{Implementación de la Elevación del Terreno}
%Solucion particular
\subsection*{Patrón Fachada}
%Para poder utilizar todos los servicios web de alturas que hagan falta
\subsection*{Cache de alturas}
%Para mejorar la latencia de datos
\section*{Open Street Maps}
\subsection*{Parseando XML}
%Leer y obtener informacion
\subsubsection*{Filtro de información}
%Quedarnos con la información útil
\subsection*{Dibujar Calles}
%De coordenadas a tiles
\subsection*{Rellenar Figuras}
%Lineas poligonales cerradas y members
\subsubsection*{Lineas poligonales cerradas}
%los parques y contornos de edificios
\subsubsection*{Relations}
%rios, mares ....
\section*{Generador de KML}
\subsection*{Multiples Escenarios}
%Soporte para varios escenarios
\subsection*{Secuencia de Tiempo}
%Soporte para la temporalidad de los eventos
\subsubsection*{Sincronización}
%Hablar de la necesidad de un agente reloj.
\subsection*{KmlPolygon}
%Agrupar poligonos por altura
\subsubsection*{Color}
%decidimos usar hexagonos de distintos colores
\subsubsection*{Transparencia}
%para dar sensacion de profundidad
\section*{Visor Bidimensional}
\subsection*{Multiples escenarios}
\section*{Estadísticas}
\subsection*{Multiples Escenarios}
%Soporte para varios escenarios
\subsection*{Ficheros CVS}
%Es lo mas comodo para las estadisticas