% Metodología e implementación

\chapter*{Implementación} \label{cap5}
\addcontentsline{toc}{chapter}{Implementación}

\pagenumbering{arabic}

\begin{flushright}
\begin{minipage}{7.85cm}
    {\em Blah blah blah.} \\  Alguien
\end{minipage}
\end{flushright}

\vspace*{5mm}

\section*{Metodología}
%Es un proyecto grande, que hay que organizarse, que usamos una
%metodologia agil, poniendonos metas a corto plazo y repartiendo la carga de
%trabajo
\subsection*{Corto Plazo}%TODO cambiarle el nombre
%Desarrollar pequeñas aplicaciones que sean funcionales y con un limite de
%tiempo, permite tener una version incremental del proyecto y facilita la
%planificación, la priorización de tareas y el reparto.
\subsection*{Forja}
%La forja esta bien, te permite controlar las tareas abiertas y el tiempo que te
%tiras con ellas, tambien vale para repartirlas
\subsection*{Blog}
%Es una buena idea, porque al escribir articulos sirve para pensar mas en lo que
%haces y darle visiblidad, tambien era un requisito para el concurso de software
%libre y sirve para historico y desarrollo  del proyecto.
\subsection*{Control de versiones}
%Es lo mejor para el trabajo distribuido, y mantiene la cohesion de lo echo
\subsubsection*{Subversion}
%Esta bien, pero es muy limitao
\subsubsection*{Git}
%Nueva forma de ver las cosas, muchos commits, cambios muy controlaos,
%posibilidad de hacer ramas y mantener siempre una rama estable.