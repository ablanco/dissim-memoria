\chapter{Introducción} \label{cap0a}
\addcontentsline{toc}{chapter}{Introducción}

\pagenumbering{arabic}

\begin{flushright}
\begin{minipage}{7.85cm}
    {\em Si la única herramienta que se tiene es un martillo, se pensará que
    cada problema que surge es un clavo.} \\  Mark Twain
\end{minipage}
\end{flushright}

\vspace*{5mm}

\section{Inicios del Proyecto}

Este proyecto empezó siendo un simulador de movimientos de refugiados ante
amenazas de grupos armados. Para poder llevar a cabo este tipo de simulación,
era necesario buscar, entre montañas de información, algo relativo a movimientos
humanos relacionados con campos de refugiados\cite{John08}.

Pero también necesitábamos más cosas, como información sobre las necesidades
alimentarias o el desplazamiento medio de un grupo de refugiados, además de
tener en cuenta factores meteorológicos, geográficos, políticos y conflictos
cercanos que pudieran modificar las rutas.

Por ello, más que un problema de simulación en sí, nos encontramos con un
problema de falta de documentación. Aunque la documentación relativa a este tipo
de desastre humanitario había crecido considerablemente en los últimos años,
no se llegaba al nivel de detalle suficiente, o al menos la información pública
y accesible no tenía dicho nivel.

Así que nos vimos obligados a abandonar dicha idea, y redirigir el proyecto a
otro tipo de catástrofes más documentadas, como por ejemplo inundaciones.

\section{Motivación}

Los desastres naturales provocan enormes pérdidas, tanto personales como
materiales. Se hace un gran esfuerzo en preparar medidas y se realizan grandes
inversiones en material, y aún así siguen produciéndose tragedias cada vez que
una inundación, un terremoto, etc, azota a algún país\cite{Cho07}.

Por ello resulta de vital importancia ser capaces de simular estos escenarios,
para poder comprobar los efectos del desastre y tomar las medidas adecuadas. El
conocimiento previo al desastre es la clave, y la única forma de obtenerlo es
la simulación.

Pero por desgracia realizar una simulación realista mediante métodos
tradicionales resulta extremadamente complejo, además de caro
computacionalmente. Por ello nosotros proponemos un simulador basado en agentes.

\section{Solución}

Los sistemas multiagente son una técnica de inteligencia artificial que nos
permite abordar este problema desde un nuevo punto de vista. Un agente es una
entidad {\em inteligente} y autónoma, pero relativamente simple, y por lo tanto
computacionalmente ligera. Es la conducta combinada de estos agentes la produce
un resultado en conjunto {\em inteligente}, un comportamiento emergente que no
tiene por qué estar planeado desde un principio.

Con esta herramienta es posible diseñar un simulador cuya carga computacional
no sea inabarcable y que produzca resultados satisfactorios.

\subsection{Objetivos}

Se pretende con este proyecto desarrollar la base de un {\bf Simulador de
Desastres}, centrándose en el caso de una inundación. El simulador ha de cumplir
las siguientes condiciones:

\begin{itemize}
 \item Ha de simular utilizando datos reales del terreno simulado. No tendría
sentido realizar una simulación si no se puede aprovechar la información
obtenida con ella.
 \item Ha de ser escalable y distribuido. Es imprescindible que se pueda
simular sobre áreas tan grandes como se quiera, y que simplemente añadiendo más
máquinas a la infraestructura de simulación esto sea posible.
 \item Ha de ser fácilmente extensible a otros desastres naturales. La
arquitectura del simulador ha de estar diseñada para que esta tarea no requiera
grandes modificaciones del código existente.
 \item Ha de respetar los estándares utilizados en los sistemas multiagente. En
concreto los estándares de comunicación entre agentes establecidos por la FIPA
\footnote{Foundation for Intelligent Physical Agents:
\url{http://www.fipa.org/}}.
 \item Ha de poder generar estadísticas sobre el estado de las personas
simuladas, para poder realizar comparaciones entre simulaciones, o con datos
reales.
\end{itemize}

Como caso guía utilizaremos las inundaciones que siguieron al {\bf huracán
Katrina} en la ciudad estadounidense de {\bf Nueva Orleans}, en el año 2005.
Hemos escogido este caso concreto por la gran cantidad de documentación
disponible.

\subsection{Resultados}

Para la visualización de resultados se ha optado por dos métodos. Un visor
bidimensional que muestra la simulación en ``tiempo real'', y un generador de
KML\footnote{Keyhole Markup Language:
\url{http://www.opengeospatial.org/standards/kml/}}. KML es un estándar para la
representación de información geográfica, y se puede abrir con múltiples
visores, uno de los más conocidos es {\em Google Earth}.

Generar ficheros KML nos permitirá mostrar los resultados de la simulación
sobre imágenes satélite del terreno simulado, con gráficos en 3D del terreno y
con los edificios modelados también en 3D. De esta manera la representación de
los resultados estará mucho más cercana a la realidad y será más accesible y
sencilla de interpretar.

\section{Software Libre}

Citando a Richard M. Stallman: {\em ``Las obras de conocimiento deben ser
libres, no hay excusas para que no sea así''}. Y de hecho no las hay, nosotros
no concebimos este proyecto de ninguna forma que no sea Software Libre.

El Software Libre es una cuestión de libertad: las personas deberían ser libres
para usar el software de todas las maneras que sean socialmente útiles. El
software difiere de los objetos materiales (como las sillas, los bocadillos o la
gasolina) en el hecho de que puede copiarse y modificarse mucho más fácilmente.
Estas posibilidades hacen al software tan útil como es; y creemos que los
usuarios de software deberían ser capaces de aprovecharlas.

Por Software Libre entendemos aquel que cumple las siguientes cuatro libertades:

\begin{enumerate}
\setcounter{enumi}{-1}
\item La libertad de usar el programa, con cualquier propósito.
\item La libertad de estudiar cómo funciona el programa y modificarlo,
adaptándolo a tus necesidades.
\item La libertad de distribuir copias del programa, con lo cual puedes ayudar a
tu prójimo.
\item La libertad de mejorar el programa y hacer públicas esas mejoras a los
demás, de modo que toda la comunidad se beneficie.
\end{enumerate}

Y persiguiendo estos objetivos utilizaremos la licencia \hyperref[ap1]{GPL}
(Licencia Pública General de GNU), en su versión 3 o posterior, para nuestro
proyecto.

\section{Estructura de la memoria}

Esta memoria está dividida en los siguientes capítulos:

\begin{description}
 \item[Simulación basada en agentes:] Donde se describen las características de
los sistemas multiagentes.
 \item[Planteamiento del problema:] Donde se describen las dificultades que hay
que abordar a la hora de simular una inundación.
 \item[Solución propuesta:] Donde se detalla el diseño del simulador
desarrollado.
 \item[Tecnología elegida:] Donde se habla de JADE y otros aspectos
técnicos del simulador.
 \item[Metodología e implementación:] Donde se explica cómo ha sido desarrollado
el proyecto.
 \item[Resultados:] Donde se muestran los resultados obtenidos al simular con
este proyecto.
 \item[Conclusiones y trabajo futuro:] Donde se comentan las ideas obtenidas y
las que quedan por desarrollar.
 \item[Apéndices:] Donde se encuentra el contenido adicional de la memoria.
  \subitem Licencia del Simulador
  \subitem IV Concurso Universitario de Software Libre
\end{description}

%%% Local Variables:
%%% mode: latex
%%% TeX-master: "../dissim"
%%% End: