% Introducción
%     - Motivación
%     - Solución
%     - Estructura de la memoria

\chapter*{Introducción} \label{cap0}
\addcontentsline{toc}{chapter}{Introducción}

\pagenumbering{arabic}

\begin{flushright}
\begin{minipage}{7.85cm}
    {\em Si la única herramienta que se tiene es un martillo, pensará que cada
    problema que surge es un clavo.} \\  Mark Twain
\end{minipage}
\end{flushright}

\vspace*{5mm}

\section*{Motivación}

Los desastres naturales provocan enormes pérdidas, tanto personales como
materiales. Se hace un gran esfuerzo en preparar medidas y se realizan grandes
inversiones en material, y aún así siguen produciéndose tragedias cada vez que
una inundación, un terremoto, etc, azota a algún país.

Por ello resulta de vital importancia ser capaces de simular estos escenarios,
para poder comprobar los efectos del desastre y tomar las medidas adecuadas. El
conocimiento previo al desastre es la clave, y la única forma de obtenerlo es
la simulación.

Pero por desgracia realizar una simulación realista mediante métodos
tradicionales resulta extremadamente complejo, además de caro
computacionalmente. Por ello nosotros proponemos un simulador basado en agentes.

\section*{Solución}

Los sistemas multi-agentes son una técnica de inteligencia artificial que nos
permiten

\section*{Estructura de la memoria}

\begin{enumerate}
 \item Simulación basada en agentes
 \item Planteamiento del problema
 \item Solución propuesta
 \item Tecnología elegida
 \item Metodología e implementación
 \item Resultados
 \item Conclusiones y trabajo futuro
\end{enumerate}