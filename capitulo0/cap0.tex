% Introducción
%     - Motivación
%     - Solución
%     - Estructura de la memoria

\chapter*{Introducción} \label{cap0}
\addcontentsline{toc}{chapter}{Introducción}

\pagenumbering{arabic}

\begin{flushright}
\begin{minipage}{7.85cm}
    {\em Si la única herramienta que se tiene es un martillo, pensará que cada
    problema que surge es un clavo.} \\  Mark Twain
\end{minipage}
\end{flushright}

\vspace*{5mm}

\section*{Motivación}

Los desastres naturales provocan enormes pérdidas, tanto personales como
materiales. Se hace un gran esfuerzo en preparar medidas y se realizan grandes
inversiones en material, y aún así siguen produciéndose tragedias cada vez que
una inundación, un terremoto, etc, azota a algún país.

Por ello resulta de vital importancia ser capaces de simular estos escenarios,
para poder comprobar los efectos del desastre y tomar las medidas adecuadas. El
conocimiento previo al desastre es la clave, y la única forma de obtenerlo es
la simulación.

Pero por desgracia realizar una simulación realista mediante métodos
tradicionales resulta extremadamente complejo, además de caro
computacionalmente. Por ello nosotros proponemos un simulador basado en agentes.

\section*{Solución}

Los sistemas multi-agentes

\section*{Estructura de la memoria}

Lorem ipsum dolor sit amet, consectetur adipiscing elit. Pellentesque lectus
nisl, hendrerit quis tempor eu, ornare placerat arcu. Pellentesque volutpat nibh
neque, sed laoreet tellus.

Donec consectetur massa at ipsum iaculis porta. Quisque dictum enim id ipsum
porttitor euismod non sed lectus.

Morbi vitae enim sem, ac imperdiet nulla. Praesent vehicula massa eu orci
elementum facilisis. Suspendisse nunc neque, facilisis in sollicitudin id,
malesuada sed elit. Vivamus id neque nisi. Nulla vehicula accumsan nisl, egestas
commodo libero ultrices eu. Cras non odio magna.