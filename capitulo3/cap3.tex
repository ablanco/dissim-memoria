% Solución propuesta (diseño)

\chapter*{Solución propuesta} \label{cap3}
\addcontentsline{toc}{chapter}{Solución propuesta}

\pagenumbering{arabic}

\begin{flushright}
\begin{minipage}{7.85cm}
    {\em No podemos resolver los problemas utilizando la misma manera de pensar
    que utilizamos cuando los creamos.} \\ Albert Einstein
\end{minipage}
\end{flushright}

\vspace*{5mm}

\section*{Definición del Escenario de Simulación}

% TODO

\section*{Agentes en la Simulación}

% TODO esquema de agentes

\subsection*{Agente Creador}

La función del agente Creador es la de lanzar la simulación. Esto lo
convierte
en un agente un poco particular, pues no tiene un papel activo en la
simulación.

El ciclo de vida de este agente será el siguiente:
\begin{enumerate}
 \item Procesar el {\bf Escenario} de simulación.
 \item Crear los agentes {\bf Entorno}, y esperar a que hayan obtenido datos
estáticos del terreno (tales como la altura o información sobre las calles).
 \item Crear los agentes de {\bf Entrada de Agua} y los agentes {\bf Peatón}.
 \item Crear el agente {\bf Reloj} y comenzar por tanto la simulación.
 \item Quedarse a la espera por si algún agente le solicita datos del
{\bf Escenario} de simulación.
\end{enumerate}

Como se puede comprobar, una vez que el agente Creador ha terminado de lanzar
la simulación adopta un papel pasivo.

\subsection*{Agente Reloj}

La necesidad de un agente que sincronizase la simulación no se hizo patente
hasta bien avanzado el desarrollo del simulador. El motivo se explica en
profundidad en el capítulo 5. % TODO link al capítulo en cuestión

Este agente tiene el objetivo de sincronizar al resto de agentes, y su
funcionamiento es muy sencillo. Dado un periodo de tiempo, definido en el
{\bf Escenario} de simulación, el agente Reloj ha de enviar regularmente un
mensaje al resto de agentes dando lugar a un ciclo de simulación.

Los agentes que reciben estos mensajes, y por lo tanto están sincronizados,
son: agentes {\bf Entorno}, agentes {\bf Entrada de Agua} y agentes {\bf
Peatón}.

\subsection*{Agentes Entorno}

Probablemente el agente con más responsabilidades y trabajo de todo el
simulador. Su misión es manejar la información relativa al terreno de un área
de simulación.

Para asegurar la escalabilidad del sistema ha de ser posible dividir el área a
simular entre varios agentes {\bf Entorno}, de manera que cada {\bf Entorno} se
pueda ejecutar en un núcleo diferente, o incluso en una máquina diferente. Esta
división ha de ser totalmente trasparente para los demás agentes.

Estos agentes se encargan de:

\begin{itemize}
 \item Obtener la elevación del terreno y la información relativa a calles,
puntos de interés, etc.
 \item Mover el agua de la inundación por el entorno.
 \item Recibir el agua que entre en la simulación desde un agente {\bf Entrada
de agua}.
 \item Informar a los agentes {\bf Peatón} de qué es lo que ven a su alrededor
dada su posición.
 \item Informar a los agentes {\bf Actualización} de la evolución de la
simulación.
% TODO más cosas?
\end{itemize}

\subsection*{Agentes Entrada de Agua}

Estos agentes representan una fuente de agua del sistema. Están localizados en
un punto del área de simulación, y con cada {\em tick} del {\bf Reloj} envían
un mensaje con cierta cantidad de agua al agente {\bf Entorno} correspondiente.

La cantidad de agua de cada mensaje dependerá del ritmo de entrada de agua en
ese punto, y del ritmo de los {\em ticks} de {\bf Reloj}.

\subsection*{Agentes Peatón}

\subsection*{Agentes Actualización}

\section*{Elevación del Terreno}

\section*{Open Street Maps}

\section*{Generador de KML}

\section*{Visor Bidimensional}