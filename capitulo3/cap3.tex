% Solución propuesta (diseño)

\chapter*{Solución propuesta} \label{cap3}
\addcontentsline{toc}{chapter}{Solución propuesta}

\pagenumbering{arabic}

\begin{flushright}
\begin{minipage}{7.85cm}
    {\em No podemos resolver los problemas utilizando la misma manera de pensar
    que utilizamos cuando los creamos.} \\ Albert Einstein
\end{minipage}
\end{flushright}

\vspace*{5mm}

\section*{Definición del Escenario de Simulación}

\section*{Agentes en la Simulación}

% TODO esquema de agentes

\subsection*{Agente Creador}

La función del agente Creador es la de lanzar la simulación. Esto lo
convierte
en un agente un poco particular, pues no tiene un papel activo en la
simulación.

El ciclo de vida de este agente será el siguiente:
\begin{enumerate}
 \item Procesar el {\bf Escenario} de simulación.
 \item Crear los agentes {\bf Entorno}, y esperar a que hayan obtenido datos
estáticos del terreno (tales como la altura o información sobre las calles).
 \item Crear los agentes de {\bf Entrada de Agua} y los agentes {\bf Peatón}.
 \item Crear el agente {\bf Reloj} y comenzar por tanto la simulación.
 \item Quedarse a la espera por si algún agente le solicita datos del
{\bf Escenario} de simulación.
\end{enumerate}

Como se puede comprobar, una vez que el agente Creador ha terminado de lanzar
la simulación adopta un papel pasivo.

\subsection*{Agente Reloj}

\subsection*{Agentes Entorno}

\subsection*{Agentes Entrada de Agua}

\subsection*{Agentes Peatón}

\subsection*{Agentes Actualización}

\section*{Elevación del Terreno}

\section*{Open Street Maps}

\section*{Generador de KML}

\section*{Visor Bidimensional}