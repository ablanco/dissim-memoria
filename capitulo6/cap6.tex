% Resultados

\chapter*{Resultados} \label{cap6}
\addcontentsline{toc}{chapter}{Resultados}

\begin{flushright}
\begin{minipage}{7.85cm}
    {\em Si buscas resultados distintos, no hagas siempre lo mismo.} \\ Albert
    Einstein
\end{minipage}
\end{flushright}

\vspace*{5mm}

\section*{Preparando el escenario}
Todo buen simulador tiene que demostrar que lo simulado tiene una relación
directa con lo que realmente pasó.

Para ello hemos recopilado la información de la inundación provocada por el
Katrina en Nueva Orleans y para poder simular la catástrofe con los parámetros
reales.
\subsection*{Área de simulación}
El huracán Katrina afectó a una parte considerable del estado de Luisiana, pero
nosotros nos centraremos en la zona mas afectada en Nueva Orlenas
\begin{figure}[H]
 \centering
 \includegraphics[width=30mm]{figuras/cap6/NOarea1.png}
 % NOarea1.png: 1066x626 pixel, 96dpi, 28.20x16.56 cm, bb=0 0 799 469
\caption{Area de Simuñación}
\end{figure}

Como se puede ver en la figura, las zonas mas afectadas por la inundación se
podrían dividir por el lecho del río.
\begin{figure}[H]
 \centering
 \includegraphics[width=30mm]{figuras/cap6/Areas Afectadas.png}
 % NOarea1.png: 1066x626 pixel, 96dpi, 28.20x16.56 cm, bb=0 0 799 469
\caption{División en zonas del área afectada}
\end{figure}

Aqui podemos apreciarlas mas en detalle.

\begin{figure}[H]
 \centering
 \includegraphics[width=30mm]{figuras/cap6/NOarea2.png}
 % NOarea1.png: 1066x626 pixel, 96dpi, 28.20x16.56 cm, bb=0 0 799 469
\caption{Zona 1 de simulación}
\end{figure}

\begin{figure}[H]
 \centering
 \includegraphics[width=30mm]{figuras/cap6/NOarea3.png}
 % NOarea1.png: 1066x626 pixel, 96dpi, 28.20x16.56 cm, bb=0 0 799 469
\caption{Zona 2 de simulación}
\end{figure}

\begin{figure}[H]
 \centering
 \includegraphics[width=30mm]{figuras/cap6/NOarea4.png}
 % NOarea1.png: 1066x626 pixel, 96dpi, 28.20x16.56 cm, bb=0 0 799 469
\caption{Zona 3 de simulación}
\end{figure}


\subsection*{Obteniendo Alturas}
Para poder simular la inundación uno de los datos que tenemos que obtener son
las alturas, para una zona tan grande como Nueva Orleans hacen falta millones
de datos de altura, sin embargo para nuestra simulación nos hemos limitado a la
parte de nueva orleans que resultó mas afectada y con una precision de 50
metros por casilla.
\subsection*{Localización de las roturas de los diques}
De este mapa podemos ver las localizzaciones donde los diques se rompieron.

\begin{figure}[H]
 \centering
 \includegraphics[width=30mm]{figuras/cap6/Roturas de Diques.png}
 % NOarea1.png: 1066x626 pixel, 96dpi, 28.20x16.56 cm, bb=0 0 799 469
\caption{Mapa con las localizaciones de las roturas de los diques}
\end{figure}
A continuación mostramos la información con la que hemos realizado la
simulación de donde se encontraban los diques.

\begin{figure}[H]
 \centering
 \includegraphics[width=30mm]{figuras/cap6/Roturas de Diques.png}
 % NOarea1.png: 1066x626 pixel, 96dpi, 28.20x16.56 cm, bb=0 0 799 469
\caption{Mapa con las localizaciones de las roturas de los diques}
\end{figure}


\begin{figure}[H]
 \centering
 \includegraphics[width=30mm]{figuras/cap6/Roturas de Diques1.png}
 % NOarea1.png: 1066x626 pixel, 96dpi, 28.20x16.56 cm, bb=0 0 799 469
\caption{Detalle de las roturas de los diques}
\end{figure}


\begin{figure}[H]
 \centering
 \includegraphics[width=30mm]{figuras/cap6/Roturas de Diques2.png}
 % NOarea1.png: 1066x626 pixel, 96dpi, 28.20x16.56 cm, bb=0 0 799 469
\caption{Detalle de las roturas de los diques}
\end{figure}

\subsection*{Rutas de evacuación y refugios}
Refugios oficiales no hubo en la inundación, lo que se hizo fué una evacuación
completa de la ciudad, para ello, hemos marcado como objetivo tres posibles
rutas de escape para los ciudadanos.

También consideramos que los edificios de los cuales podemos obtener
información a través de OSM pudieron ser refugios de emergencia donde
temporalmente pudieron refugiarse algunos ciudadanos.

\begin{figure}[H]
 \centering
 \includegraphics[width=30mm]{figuras/cap6/Rutas de evacuacion.png}
 % NOarea1.png: 1066x626 pixel, 96dpi, 28.20x16.56 cm, bb=0 0 799 469
\caption{Rutas de evacuación}
\end{figure}

\section*{Resultados}
%echamos a correr el simulador y comentamos la jugada
\subsection*{Visualización de Resultados}
%Ventanitas del simulador o capturas de google earth
\section*{Simulación frente Realidad}
%Es cuestion de preparar un escenario de nueva orleans y comparar con la
%bibliografia

%%% Local Variables:
%%% mode: latex
%%% TeX-master: "../dissim"
%%% End: