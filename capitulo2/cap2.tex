% del problema: Simulación de inundaciones

\chapter*{Simulación de Inundaciones} \label{cap2}
\addcontentsline{toc}{chapter}{Simulación de Inundaciones}

\begin{flushright}
\begin{minipage}{7.85cm}
    {\em Sólo podemos ver un poco del futuro, pero lo suficiente para darnos
    cuenta de que hay mucho por hacer.} \\  Alan Turing
\end{minipage}
\end{flushright}

\vspace*{5mm}

\section*{Planteamiento del Problema}
%simular el agua, simular las personas, obtener información real

Nuestro simulador consta de tres grandes elementos: el agua, las personas y la
ciudad.

Para poder simular una inundación primero necesitamos saber como se comporta el
agua. Desde aspectos básicos como que tiende a moverse hacia las zonas bajas del
terreno, hasta cosas no tan básicas como puede ser dinámica de fluidos.

Por otro lado queremos simular la reacción que tendrían las personas ante una
inundación, para ello necesitamos saber la velocidad a la que pueden desplazarse
y los posibles cursos de acción que tomarían.

Por último queremos que nuestro sistema sea global, de tal forma que no esté
restringido sólo a una zona concreta del planeta. Para ello hemos de utilizar
sistemas de información global. Aunque siempre estaremos a expensas de la
disponibilidad de los datos.

\section*{Movimiento del Agua}

Para simular el movimiento del agua hemos contemplado muchas posibles
soluciones. Puesto que aplicar mecánica de fluidos no es algo para nada trivial,
hemos optado por una simplificación, de forma que el simulador no se vuelva
computacionalmente pesado.

Nuestra intención no es crear un simulador del comportamiento del agua
fidedigno a la realidad a pequeña escala. No requerimos de exactitud
milimétrica, si no que nos podemos permitir trabajar de una manera más
simplista y a gran escala.Tampoco nos preocupamos de si los elementos que
arrastra la inundación modifican su trayectoria o comportamiento, puesto que
este problema ya sería en sí el tema de una tesis completa.

En el simulador abordamos dos de los problemas del movimiento del agua, la
dirección y la velocidad de propagación.

\subsection*{Dirección del Agua}

Cuando se simula una inundación es vital saber en qué dirección se moverá la
masa de agua, puesto que queremos conocer cuáles serán los terrenos afectados
por la inundación.

La dirección de movimiento del agua depende de muchos factores, el más
determinante de ellos es, debido a la energía potencial, la altura. El agua
siempre tiende a ir a zonas más bajas.

Otro factor que influye en la dirección del agua es la fuerza en cada momento
de ésta y su velocidad, dado que puede llegar a salvar obstáculos e inundar
terrenos, que sin tener en cuenta estos factores, nunca hubieran sido inundados.

Para resolver este problema nosotros sólo disponemos de la altura del terreno
como único dato.

\subsection*{Velocidad de Propagación del Agua}

La velocidad de propagación de un fluido en un entorno irregular es un problema
muy complejo, nos hemos visto obligados a simplificarlo considerablemente.
Habría que tener en cuenta variables tales como la fuerza inicial del agua,
dirección, presión, la ganancia o perdida de energía al bajar por un terreno o
subir, el posible desgaste del terreno o incluso su resistencia al paso del
agua, etc.

El problema no es que no seamos capaces de simular este sistema con nuestros
agentes, o que sea directamente inabordable por su complejidad, si no que no
tenemos datos para poder recrear el escenario necesario. Datos sobre la
composición del terreno, los obstáculos, la resistencia de los materiales,
su rugosidad y permeabilidad al agua, etc. No se encuentran tan fácilmente.

\subsection*{Otros Problemas con el Agua}

Entre otros muchos posibles problemas que podríamos haber abordado, dada su
sencillez y facilidad de implementación, y gracias a la versatilidad de nuestro
sistema, están la absorción de agua por parte del terreno y las filtraciones de
agua.

La absorción de agua podría no ser despreciable, pero por otra parte cuando se
produce una inundación es porque el terreno ya está saturado de agua y, por lo
tanto, el agua no absorbida se queda en la superficie. % TODO Citar algo
También podrían producirse filtraciones que podrían provocar movimientos de
agua por debajo de la tierra, llegando a zonas bajas sin que aparezca una
corriente en la superficie. % TODO Citar algo

Pero una vez más, por falta de datos, se han tenido que quedar en el tintero.

\section*{Inundaciones en un Entorno Urbano}

Simular una inundación en un entorno urbano tiene sus propios problemas, tales
como el papel de los ríos, las calles, la forma de los edificios, la red de
alcantarillado, etc.

\subsection*{Ríos}

En una ciudad, un río será, por lo general, un sector conflictivo cuando se
produzca una inundación, pues probablemente sea el causante de la inundación.

Los cálculos necesarios para calcular el caudal introducido y el caudal evacuado
son complejos de por sí\cite{Kollinger}, pero podrían ser resueltos por nuestro
sistema multiagente si tuviésemos acceso a los datos necesarios, tales como la
forma del lecho del río, etc.

\subsection*{Forma de los Edificios}

La forma de los edificios también va a influir en el desarrollo de la
inundación. Cuando entre el agua dentro del edificio se modificará su velocidad
y trayectoria, y además las paredes o fachadas pueden canalizar el agua y llegar
a formar piscinas, o inundar sótanos.

Sin conocer la forma del edificio no podemos hacer una simulación del todo fiel
a realidad, pues estos afectan al movimiento del agua y evolución de la
inundación. Con una fuente de datos de elevación suficientemente buena, que
contenga las alturas de los edificios, el simulador será capaz de simular
adecuadamente el agua. Pero una vez más, estamos a merced de la calidad de los
datos.

\subsection*{Calles y Carreteras}

Las calles y carreteras tienen la particularidad de ser prácticamente planas, y
de inclinación suave, lo que las hacen idóneas para que el agua pueda moverse
sin mucha dificultad. Actúan como canales conduciendo el agua a más velocidad
por la ciudad que el agua que atraviesa edificios.

\subsection*{Alcantarillado}

Todas las ciudades tienen una red de alcantarillado, al menos todas las del
primer mundo, que no es mas que un conjunto de canales subterráneos dedicados
al desplazamiento de agua.

En caso de inundación, las redes del alcantarillado moverán una gran cantidad de
agua que no se verá afectada por obstáculos o irregularidades del terreno. Es un
problema que se puede asemejar al de las filtraciones, pero con un factor de
incidencia mucho más alto.

Para poder incluirlo en el simulador se hace necesario el disponer de un mapa
del alcantarillado y de sus zonas de desagüe. Se podría asumir que prácticamente
debajo de cada calle habrá una conducción de alcantarilla, pero se estarían
ignorando los caudales de las conducciones y los puntos de entrada o salida de
agua.

\subsection*{Prioridades de las personas}

Se han realizado estudios\cite{Liu07} en algunas poblaciones sobre qué es
lo que harían sus habitantes en caso de crisis. Se han considerado aspectos
tales como su disponibilidad a salvar a familiares en apuros, o a vecinos, o
simplemente a cualquier desconocido que lo necesite.

En el simulador esto se podría traducir en comportamientos de ayuda entre
agentes, como indicarles el camino a seguir o incluso acompañarles a un
lugar seguro.

\subsection*{Supervivencia en un entorno inundado}

También es importante, ante una catástrofe, recoger información acerca de los
supervivientes para posteriores análisis. Por ejemplo, tras las inundaciones
producidas por el huracán Katrina en Nueva Orleans se han realizado
encuestas\cite{Washington05} de este tipo.

Información sobre los peores momentos de la inundación, las zonas más afectadas,
los lugares de mayor concentración de personas, etc. Este tipo de datos pueden
darnos una idea aproximada de cómo se podrían prevenir ciertas situaciones de
riesgo. También nos puede ser útil para realizar un informe de daños y
necesidades de la población tras el desastre.

En nuestro caso contar con este tipo de informaciones nos permitirá contrastar
los resultados simulados con casos reales, para realizar comparaciones y sacar
conclusiones.

\subsection*{Evacuación de personas}

La correcta evacuación de una ciudad ante un desastre es algo muy importante, y
de ello dependen las vidas de muchas personas. La rapidez y eficacia de la
evacuación puede determinar la magnitud de las pérdidas en vidas humanas. Por
ello, poder simular una evacuación supone una gran ventaja dado que nos
aportaría datos sobre rutas frecuentes de evacuación\cite{Lammel08}, o cómo se
comportarían las personas en situaciones de pánico\cite{Chu05}, etc.

Con esta información se podrían concentrar los esfuerzos de las autoridades en
los puntos críticos. Manteniendo las vías principales abiertas, o en su defecto,
proporcionando caminos alternativos y demás soluciones a los posibles problemas.

\section*{Obtener información real sobre los terrenos}

Lo verdaderamente interesante de nuestro proyecto es poder probarlo sobre
lugares reales, para ello es necesario obtener la información, pero además,
decidimos hacerlo de tal forma de que el lugar no fuera el problema, siempre que
tengamos datos de ese lugar, podremos integrarlos en nuestro simulador y simular
el problema allí donde se encuentre. Para ello necesitábamos usar un sistema de
geolocalización y servidores de información globales.

Sin embargo para los servidores de información lo teníamos mas difícil al no
existir servicios a nivel global, al menos, disponibles al publico.

\subsection*{Elevación del terrenos}

No existe, o al menos no conocemos un servidor global de alturas.

Por lo que de momento, solo disponemos de un servicio de alturas para el caso de
nuestro estudio, ''Nueva Orleans''.

Si embargo, como ya hemos comentado antes, nuestra pretensión es que la
simulación se pueda realizar en cualquier parte del planeta, por lo que la
nuestro sistema esta preparado para poder añadirle fácilmente otros servidores a
fin de poder simular en cualquier otro lugar. 

\subsection*{Posibles rutas de evacuación}

Con nuestro simulador no solo pretendemos simular la inundación en sí, sino la
posible reacción de las personas ante la inundación, para ello, como ya hemos
visto antes, necesitamos saber que es lo que las personas harían ante una
crisis y además, conocer la ciudad en sí en la que se desarrolla, sobre todo sus
calles, puesto que es probablemente, la ruta mas probable de la ciudad.
En una evacuación ante una inundación es muy importante conocer el plano de la
ciudad, para poder registrar las posibles rutas de de evacuación.

Para obtener los planos de las ciudades optamos por ''Open Streets Map'' ya que
al ser un estándar abierto y con información de ciudades a nivel global pues
tenía todo lo que le pudiéramos necesitar. 

También contiene mucha información adicional, tales como parques, hospitales,
aeropuertos, y lugares de interés que podrían ser identificados como puntos de
salvamento, o al menos objetivos para los evacuados.

Gracias a este servicio obtenemos también la posición y forma de los ríos,
mares, puentes ... muy importantes para las inundaciones. 

Nuestro problema ahora se centraría en adaptar esos datos de forma que pudieran
ser codificados e interpretados por nuestro simulador, no solo trazando calles
y carreteras por nuestro simulador, sino reconociendo posibles puntos de
salvamento o objetivos de los evacuados.

\section*{Geolocalización}

Nosotros queremos simular en terrenos reales, y los terrenos vienen
localizados por coordenadas geográficas (WGS84).

Gracias a las coordenadas podemos obtener información geolocalizada de
cualquier parte del mundo e identificarla con esa zona.

Por lo que nos encontramos con el problema de la discretización de los
datos y la inevitable perdida de información.

\subsection*{Discretización de los Datos}
Muchas veces, debido a que nosotros trabajamos sobre un mapa hexagonal, al
hacer el cambio desde coordenadas geográficas, dependiendo del nivel de
precisión de nuestra rejilla, inevitablemente vamos a sufrir una perdida de
información.

Otras veces los datos recibidos de las localizaciones no son del todo exactos o
precisos, como por ejemplo las localizaciones de los edificios, que vienen
representadas por el centro del edificio, y no por su perímetro, por lo que a
veces nos veremos obligados a aproximar puntos o a cambiarlos un poco de sitio.

Por ello pueden aparecer calles de un tamaño mayor al real o incluso un poco
desplazadas, pero suponemos que eso no tendrá consecuencias mayores.

\section*{Visualización de los Resultados}

Nuestra intención no es solo simular una inundación, si no que se pueda ver,
paso a paso, la evolución de esta misma. 

Esto conlleva los siguiente:
\begin {itemize}
\item Necesidad de conocer en que paso de la simulación nos encontramos.
\item Llevar un historial de la inundación.
\item Realizar la inundación ordenadamente.
\end {itemize}

\subsection*{Visor Simple}
Para las primeras fases del desarrollo y para poder ver de una forma simple y
sin ayuda de programas externos, necesitamos un visor nativo del sistema, con
el que poder seguir el progreso y comportamiento del agua y de los demás
agentes implicados en la simulación.
\subsection*{Kml}

Una vez terminada la simulación, queremos poder tener un historial de todo lo
que haya ocurrido y la facilidad de poder navegar dentro de la simulación para
poder recrear el estado de la simulación en un momento preciso.

Para ello dispondremos de un sistema que guarde la información de cada
instantánea de la simulación dentro de un fichero Kml.

Hemos elegido Kml debido a que es un estándar y a la versatilidad y potencia
gráfica que nos proporciona.

Utilizar Kml también nos obligará a tratar con problemas de geometría
computacional, y de topología. 

\subsection*{Estadísticas}
Una buena forma de comparar entre los distintos comportamientos implementados
para el agua o las personas o cualquier otro agente, son las estadísticas, así
que tendremos que crear un sistema que sea capaz de recoger estadísticas de tal
forma que nos puedan dar información importante sobre la simulación y así poder
discernir entre que métodos o estrategias son mejores con respecto a que
objetivos.

%%% Local Variables:
%%% mode: latex
%%% TeX-master: "../dissim"
%%% End: