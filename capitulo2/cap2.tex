% Planteamiento del problema: Simulación de inundaciones

\chapter*{Simulación de Inundaciones} \label{cap2}
\addcontentsline{toc}{chapter}{Simulación de Inundaciones}

\pagenumbering{arabic}

\begin{flushright}
\begin{minipage}{7.85cm}
    {\em Blah blah blah.} \\  Alguien
\end{minipage}
\end{flushright}

\vspace*{5mm}

\section*{Planteamiento del Problema}
El problema es que las cosas se inundan y eso esta jodido porque simular el movimiento del agua
es una mierda, también esta el problema de simular la gente correteando loca perdida, y bueno
como las cosas tienen que ser reales pues necesitamos información real.

\section*{El movimiento del agua}
El tema de la simulacion de fluidos esta to tensa, pero mas o menos nos las habiamos obiando la 
mayor parte de la termodinamica y asumiendo solo que el agua se mueve a los sitios mas bajos, sin
tener en cuenta ni velocidad, ni temperatura, ni presión atmosférica ni cambios en la densidad
ni na de na. Con esto esto tenemos tela de problemas, porque no sabemos cuanto se mueve el agua
en un determinado tiempo y claro, esto es un tange total (explicar bucle).

\section*{El comportamiento del las personas frente a una situación de crisis}
Pues aquí podemos explicar un poco lo de los papers, de la gente que ayudaba a los vecinos, de
cuanto podían correr con el agua al cuello, de las reacciones de irse a terrenos elevados y de 
pisotear a su vecino si va delante, cosas así.

\section*{Obtener información real sobre los terrenos}
Queremos simular cosas reales, pues necesitamos datos reales, como la altura del terreno y
los planos de una ciudad. Explicar lo del servicio web de las alturas y osm
Explicar tambien que como podemos obtener los datos de cualqueir parte del mundo, nuestro sistema
se puede portar a cualquiera ciudad del mundo y tal y tal.