% Planteamiento del problema: Simulación de inundaciones

\chapter*{Simulación de Inundaciones} \label{cap2}
\addcontentsline{toc}{chapter}{Simulación de Inundaciones}

\pagenumbering{arabic}

\begin{flushright}
\begin{minipage}{7.85cm}
    {\em Sólo podemos ver poco del futuro, pero lo suficiente para darnos cuenta
    de que hay mucho que hacer.} \\  Alan Turing
\end{minipage}
\end{flushright}

\vspace*{5mm}

\section*{Planteamiento del Problema}
%simular el agua, simular las personas, obtener información real

Nuestra simulación se basa en tres grandes pilares. El agua, las personas y la
ciudad. 

Para poder simular una inundación primero necesitamos saber como se comporta el
agua, cosas básicas como que tiende a moverse hacia las zonas bajas del terreno
hasta cosas no tan básicas como dinámica de fluidos.

Por otro lado queremos simular la reacción que tendrían las personas ante una
inundación, por ello necesitamos saber desde la velocidad a la que pueden
desplazarse, hasta sus posibles reacciones ante la llegada de una inundación.

Por ultimo queremos que nuestro sistema sea global, de tal forma que no este
restringido solo a una zona concreta del planeta, por ello hemos utilizado
sistemas de información global, como por ejemplo ``Open Streets Maps``, o en el
caso de estados unidos, un servidor de alturas. Aunque siempre estemos a
expensas de la disponibilidad de los datos, siempre que los haya y sean
accesibles, los podremos utilizar.

\section*{El movimiento del agua}

Para simular el movimiento del agua, hemos contemplado muchas posibles
soluciones a la simulación, puesto que simular fluidos y mecánica de fluido no
es algo para nada trivial, nosotros hemos optado por una simplificación para
poder abordar el tema del movimiento del agua. Puesto que nuestro sistema no
esta pensado para trabajar en tiempo real, sino a intervalos de tiempo, no
estamos obligados a utilizar un modelo muy estricto del agua, por lo que optamos
por varias simplificaciones.

\subsection*{Mecánica de fluidos}

Como ya hemos comentado anterior mente nuestra intención no es hacer un sistema
de agentes para simular grandes cantidades de fluidos moviéndose por un terreno
irregular, con obstáculos y con cálculos exactos de velocidades. Tampoco nos
preocupamos de si los elementos que arrastra la inundación modifican su
trayectoria o comportamiento, puesto que este problema, ya sería en si el tema
de una tesis completa.

En esta simulación abordamos dos de los problemas del movimiento del agua, la
dirección y la velocidad de propagación.

\subsubsection*{Dirección del Agua}

Para resolver el problema de la posible dirección que tomaría una inundación,
hemos asumido la hipótesis de que el agua siempre tenderá a propagarse desde un
terreno elevado, a otro menos elevado.

\subsubsection*{Velocidad de propagación del Agua}

Para abordar el problema de la velocidad de propagación nos hemos visto
obligados a simplificar muchísimo el sistema, puesto que tener en cuenta
variables tales como la fuerza inicial del agua y dirección, la presión, la
ganancia o perdida de fuerza al bajar por un terreno o subir, el posible
desgaste del terreno o incluso su resistencia a paso del agua, al tener mas o
menos obstáculos que pudieran oponerse y frenar el agua. Todo esto no hemos
podido abordarlo, no solo por su complejidad, sino por la falta de datos para
esta simulación. Datos sobre la composición del terreno, los obstáculos,
resistencias de materiales, dureza, permeabilidad al agua ... no se encuentran
tan fácilmente.

Por lo tanto, vistas nuestras limitaciones y sobre todo nuestra intención de un
modelo simple, a la hora de simular la propagación de una cierta cantidad de
agua sobre un terreno nos apoyamos en la iteración sobre el bucle del
comportamiento del agua, asumimos (que ya es mucho asumir), que en cada paso del
bucle, el agua se mueve hacia cuadrado adyacente que tenga una altura menor que
la propia, por lo tanto, el agua se moverá a tantas casillas adyacentes como
vueltas de el bucle. Así es de una manera muy simple, como simulamos la
propagación, dándole un valor de repetición a ese bucle.

Para el cálculo de este parámetro ............ bla bla bla

De esta forma tan simple, podemos controlar de una cierta manera razonable la
velocidad de propagación que tendrá el agua.

\subsubsection*{Otros problemas}

Entre otros muchos posibles problemas que podríamos haber abordado por su
sencillez y por la versatilidad de nuestro sistema ya que el coste de
implementación habría sido muy bajo son, entre otros el problema de la absorción
de agua por parte del terreno, que podría no ser despreciable y por supuesto el
tema de las filtraciones de agua, puesto que el agua también podría moverse por
debajo de la tierra y llegar a zonas mas bajas sin tener contacto directo. Pero
una vez mas por falta de datos, se han tenido que quedar en el tintero.

\section*{Personas frente a una situación de crisis}

Pues aquí podemos explicar un poco lo de los papers, de la gente que ayudaba a
los vecinos, de cuanto podían correr con el agua al cuello, de las reacciones de
irse a terrenos elevados y de pisotear a su vecino si va delante, cosas así.

\subsection*{Prioridades de las personas}

Asta que no ven el agua venir no se Las personas quieren salvarse, salvar a su
familia y si pueden, a alguien mas

\subsection*{Supervivencia en un entorno inundado}

Pues eso, mas y menos de la vida con el agua al cuello

\subsection*{Acciones contra la catástrofe}

\section*{Obtener información real sobre los terrenos}
%obtención de información global

Lo verdaderamente interesante de nuestro proyecto es poder probarlo sobre
lugares reales, para ello es necesario obtener la información, pero además,
decidimos hacerlo de tal forma de que el lugar no fuera el problema, siempre que
tengamos datos de ese lugar, podremos integrarlos en nuestro simulador y simular
el problema allí donde se encuentre. Para ello necesitábamos usar un sistema de
geolocalización y servidores de información globales.

Sin embargo para los servidores de información lo teníamos mas difícil al no
existir servicios a nivel global, al menos, disponibles al publico.

\subsection*{Elevación del terrenos}

No existe, o al menos no conocemos un servidor global de alturas.

Por lo que de momento, solo disponemos de un servicio de alturas para el caso de
nuestro estudio, Nueva Orleans.

Si embargo, como ya hemos comentado antes, nuestra pretensión es que la
simulación se pueda realizar en cualquier parte del planeta, por lo que la
nuestro sistema esta preparado para poder añadirle fácilmente otros servidores a
fin de poder simular en cualquier otro lugar. 

\subsection*{Posibles rutas de evacuación}

En una evacuación ante una inundación es muy importante conocer el plano de la
ciudad, para poder registrar las posibles rutas de de evacuación.
Para obtener los planos de las ciudades optamos por ''Open Streets Map'' ya que
al ser un estándar abierto y con información de ciudades a nivel global pues
tenía todo lo que le pudiéramos necesitar. 
También contiene mucha información adicional, tales como parques, hospitales,
aeropuertos, y lugares de interés que podrían ser identificados como puntos de
salvamento, o al menos objetivos para los evacuados.
Gracias a este servicio obtenemos también la posición y forma de los ríos,
mares, puentes ... muy importantes para las inundaciones. 

\subsection*{geolocalización}

Resolver la geolocalización fue trivial, usamos coordenadas geográficas (WGS84).
Ya que prácticamente toda la información de geolocalización esta en este sistema
al ser un estándar desde hace mucho tiempo.
