% del problema: Simulación de inundaciones

\chapter*{Simulación de Inundaciones} \label{cap2}
\addcontentsline{toc}{chapter}{Simulación de Inundaciones}

\pagenumbering{arabic}

\begin{flushright}
\begin{minipage}{7.85cm}
    {\em Sólo podemos ver un poco del futuro, pero lo suficiente para darnos
    cuenta de que hay mucho por hacer.} \\  Alan Turing
\end{minipage}
\end{flushright}

\vspace*{5mm}

\section*{Planteamiento del Problema}
%simular el agua, simular las personas, obtener información real

Nuestra simulación se basa en tres grandes pilares. El agua, las personas y la
ciudad. 

Para poder simular una inundación\cite{simulator} primero necesitamos saber
como se comporta el
agua, cosas básicas como que tiende a moverse hacia las zonas bajas del terreno
hasta cosas no tan básicas como dinámica de fluidos.

Por otro lado queremos simular la reacción que tendrían las personas ante una
inundación, por ello necesitamos saber desde la velocidad a la que pueden
desplazarse, hasta sus posibles reacciones ante la llegada de una inundación.

Por ultimo queremos que nuestro sistema sea global, de tal forma que no este
restringido solo a una zona concreta del planeta, por ello hemos utilizado
sistemas de información global, como por ejemplo ''Open Streets Maps'', o en
el caso de estados unidos, un servidor de alturas. Aunque siempre estemos a
expensas de la disponibilidad de los datos, siempre que los haya y sean
accesibles, los podremos utilizar.

\section*{El movimiento del agua}

Para simular el movimiento del agua, hemos contemplado muchas posibles
soluciones a la simulación, puesto que simular fluidos y mecánica de fluido no
es algo para nada trivial, nosotros hemos optado por una simplificación para
poder abordar el tema del movimiento del agua. Puesto que nuestro sistema no
esta pensado para trabajar en tiempo real, sino a intervalos de tiempo, no
estamos obligados a utilizar un modelo muy estricto del agua, por lo que optamos
por varias simplificaciones.

\subsection*{Mecánica de fluidos}

Como ya hemos comentado anterior mente nuestra intención no es hacer un sistema
de agentes para simular grandes cantidades de fluidos moviéndose por un terreno
irregular, con obstáculos y con cálculos exactos de velocidades. Tampoco nos
preocupamos de si los elementos que arrastra la inundación modifican su
trayectoria o comportamiento, puesto que este problema, ya sería en si el tema
de una tesis completa.

En esta simulación abordamos dos de los problemas del movimiento del agua, la
dirección y la velocidad de propagación.

\subsubsection*{Dirección del Agua}

Cuando se produce una inundación, es importante saber en que dirección se
propagará la masa de agua, puesto que necesitamos saber cuales serán los
terrenos afectos por la inundación.

La dirección del agua dependen de muchos factores, el mas determinante de ellos
es, debido a la energía potencial, la altura, el agua siempre va a tender a ir
desde donde se encuentra al punto mas bajo accesible.

Otro factor que influye en la dirección del agua, aunque en menor medida es la
fuerza inicial del agua y su velocidad, puesto que pueden desviar un poco la
corriente principal e incluso salvar obstáculos o subir hasta una determinada
altura en un terreno e inundar terrenos que sin tener en cuenta estos factores
nunca hubieran sido inundados.

Para resolver este problema nosotros solo disponemos de la altura del terreno
como único dato.

\subsubsection*{Velocidad de propagación del Agua}

La velocidad de propagación de un fluido en un entorno irregular es un problema
muy complejo, nos hemos visto 
obligados a simplificar muchísimo el sistema,
puesto que tener en cuenta
variables tales como la fuerza inicial del agua, dirección, la presión, la
ganancia o perdida de fuerza al bajar por un terreno o subir, el posible
desgaste del terreno o incluso su resistencia a paso del agua, al tener mas o
menos obstáculos que pudieran oponerse y frenar el agua. 

El problema no es que no seamos capaces de simular este sistema con nuestros
agentes, o que sea inabordable por su complejidad, sino que además no tenemos 
datos para poder recrear el escenario
esta simulación. Datos sobre la composición del terreno, los obstáculos,
resistencias de materiales, dureza, permeabilidad al agua ... no se encuentran
tan fácilmente.

%Esto ya forma parte de las explicaciones
Por lo tanto, vistas nuestras limitaciones y sobre todo nuestra intención de un
modelo simple, a la hora de simular la propagación de una cierta cantidad de
agua sobre un terreno nos apoyamos en la iteración sobre el bucle del
comportamiento del agua, asumimos (que ya es mucho asumir), que en cada paso del
bucle, el agua se mueve hacia cuadrado adyacente que tenga una altura menor que
la propia, por lo tanto, el agua se moverá a tantas casillas adyacentes como
vueltas de el bucle. Así es de una manera muy simple, como simulamos la
propagación, dándole un valor de repetición a ese bucle.

Para el cálculo de este parámetro ............ bla bla bla

De esta forma tan simple, podemos controlar de una cierta manera razonable la
velocidad de propagación que tendrá el agua.

\subsubsection*{Otros problemas con el agua}

Entre otros muchos posibles problemas que podríamos haber abordado por su
sencillez y por la versatilidad de nuestro sistema ya que el coste de
implementación habría sido muy bajo son, entre otros el problema de la absorción
de agua por parte del terreno, que podría no ser despreciable y por supuesto el
tema de las filtraciones de agua, puesto que el agua también podría moverse por
debajo de la tierra y llegar a zonas mas bajas sin tener contacto directo. Pero
una vez mas por falta de datos, se han tenido que quedar en el tintero.


\section*{Inundaciones en un entorno urbano}

Simular una inundación en un entorno urbano tiene sus propios problemas, tales
como la el papel de los ríos, las calles, la forma de los edificios, la red de
alcantarillado, etc... \cite{simulator}
\subsection*{Los ríos}
En una ciudad, un río será por lo general, un sector conflictivo cuando se
produzca una inundación, puesto que aparte de que probablemente sea el causante
de la inundación  también es una zona en la que pueden producirse.

Los calculos necesarios para calcular el caudal introducido y caudal evacuado
\cite{desvordamiento} son complejos de por si, pero podrían ser resueltos por
nuestro sistema multiagente, sin embargo el problema de la obtencion de datos,
es mucho mas complicado, puesto que en muy pocas circunstancias se tiene
constancia de los datos necesarios.

\subsection*{La forma de los edificios}
La forma de los edificios también va a influir en el desarrollo de la
inundación, no solo en que cuando entre el agua dentro del edificio modificará
su velocidad y trayectoria, sino con las paredes o fachadas, que pueden
canalizar el agua e incluso formar piscinas en su interior o inundar sótanos.

Sin conocer la forma del edificio no podemos hacer una simulación del todo fiel
a realidad puesto que algunos terrenos serán afectados de forma diferente si
hay un edificio delante de ellos que frena el avance del agua, o si por el
contrario, un edificio hace de tapón y provoca un crecimiento anómalo del nivel
del agua.

\subsection*{Las calles y carreteras}
Las calles y carreteras tienen una peculiaridad, puesto que al ser
prácticamente planas y de inclinación suave las hacen idóneas para que el agua
pueda moverse sin mucha dificultad por la ciudad. Ya que a un nivel pequeño del
agua sirven como canales que conducen en agua hacia un lugar mas bajo pero a
mucha mas velocidad que si lo hicieran a través de los edifcios o por cualquier
otra superficie no plana.

\subsection*{El alcantarillado}
Todas las ciudades tienen una red de alcantarillado, que no es mas que un
conjunto de canales que desplazan el agua desde la ciudad hasta un punto de la
ciudad. Por las redes del alcantarillado va a circular gran cantidad de agua
durante una inundación y no se verá afectada por la elevación del terreno ya
que al ser subterránea y estar canalizada va a llevar una gran cantidad de agua
a otro punto.

Este es un problema que se puede asemejar al de las filtraciones, pero con un
factor mucho mas alto.

Una vez mas habría que disponer de un mapa del alcantarillado y de sus zonas de
desagüe, aunque, se puede asumir que prácticamente debajo de cada carretera
habrá una alcantarilla, aún así se estarían ignorando muchos otros puntos de
entrada o salida de agua.

\subsection*{Prioridades de las personas}

Se han realizado estudios \cite{prioridades} sobre poblaciones acerca de que es
lo que harían en caso de crisis, tales como su disponibilidad a salvar a sus
familiares en apuros, o al vecino, o simplemente a alguien que lo necesite.
Nuestro problema podría ser simular comportamientos de ayuda entre agentes
tales como indicarles el camino a seguir o incluso acompañarles a un lugar
seguro.

\subsection*{Supervivencia en un entorno inundado}
%TODO
Pues eso, mas y menos de la vida con el agua al cuello

\subsection*{Evacuacion de personas}
%como se evacuan las personas de una ciudad
La correcta evacuacion de una ciudad ante un desastre es algo muy importante de
lo que pueden dender la vida de muchas personas, la rapidez y eficacia de la
evacuación puede determinar la magnitud de la catastrofe en vidas humanas. Por
ello, poder simular como se realizaria una simulación podría darnos datos de
rutas frecuentes de evacuacion \cite{evac08} o en situaciones de panico
\cite{panicevac} y concentrar las esfuerzos de las autoridades para mantener
estas vias abiertas, o en su defecto, detectar vias alternativas y posibles
problemas.

Otro problema a plantearse serían la evacuación de los edificios en si
\cite{kyoto}, puesto que en una gran ciudad, tan importante es la evacuacion de
las calles, como de los grandes edificios donde se encuentran la mayoría de la
gente.
 

\section*{Obtener información real sobre los terrenos}
%obtención de información global

Lo verdaderamente interesante de nuestro proyecto es poder probarlo sobre
lugares reales, para ello es necesario obtener la información, pero además,
decidimos hacerlo de tal forma de que el lugar no fuera el problema, siempre que
tengamos datos de ese lugar, podremos integrarlos en nuestro simulador y simular
el problema allí donde se encuentre. Para ello necesitábamos usar un sistema de
geolocalización y servidores de información globales.

Sin embargo para los servidores de información lo teníamos mas difícil al no
existir servicios a nivel global, al menos, disponibles al publico.

\subsection*{Elevación del terrenos}

No existe, o al menos no conocemos un servidor global de alturas.

Por lo que de momento, solo disponemos de un servicio de alturas para el caso de
nuestro estudio, ''Nueva Orleans''.

Si embargo, como ya hemos comentado antes, nuestra pretensión es que la
simulación se pueda realizar en cualquier parte del planeta, por lo que la
nuestro sistema esta preparado para poder añadirle fácilmente otros servidores a
fin de poder simular en cualquier otro lugar. 

\subsection*{Posibles rutas de evacuación}

Con nuestro simulador no solo pretendemos simular la inundación en sí, sino la
posible reacción de las personas ante la inundación, para ello, como ya hemos
visto antes, necesitamos saber que es lo que las personas harían ante una
crisis y además, conocer la ciudad en sí en la que se desarrolla, sobre todo sus
calles, puesto que es probablemente, la ruta mas probable de la ciudad.
En una evacuación ante una inundación es muy importante conocer el plano de la
ciudad, para poder registrar las posibles rutas de de evacuación.

Para obtener los planos de las ciudades optamos por ''Open Streets Map'' ya que
al ser un estándar abierto y con información de ciudades a nivel global pues
tenía todo lo que le pudiéramos necesitar. 

También contiene mucha información adicional, tales como parques, hospitales,
aeropuertos, y lugares de interés que podrían ser identificados como puntos de
salvamento, o al menos objetivos para los evacuados.

Gracias a este servicio obtenemos también la posición y forma de los ríos,
mares, puentes ... muy importantes para las inundaciones. 

Nuestro problema ahora se centraría en adaptar esos datos de forma que pudieran
ser codificados e interpretados por nuestro simulador, no solo trazando calles
y carreteras por nuestro simulador, sino reconociendo posibles puntos de
salvamento o objetivos de los evacuados.

\section*{Geolocalización}

Nosotros queremos simular en terrenos reales, y los terrenos vienen
localizados por coordenadas geográficas (WGS84).

Gracias a las coordenadas podemos obtener información geolocalizada de
cualquier parte del mundo e identificarla con esa zona.

Por lo que nos encontramos con el problema de la discretización de los
datos y la inevitable perdida de información.

\subsection*{Discretización de los Datos}
Muchas veces, debido a que nosotros trabajamos sobre un mapa hexagonal, al
hacer el cambio desde coordenadas geográficas, dependiendo del nivel de
precisión de nuestra rejilla, inevitablemente vamos a sufrir una perdida de
información.

Otras veces los datos recibidos de las localizaciones no son del todo exactos o
precisos, como por ejemplo las localizaciones de los edificios, que vienen
representadas por el centro del edificio, y no por su perímetro, por lo que a
veces nos veremos obligados a aproximar puntos o a cambiarlos un poco de sitio.

Por ello pueden aparecer calles de un tamaño mayor al real o incluso un poco
desplazadas, pero suponemos que eso no tendrá consecuencias mayores.

\section*{Visualización de los Resultados}

Nuestra intención no es solo simular una inundación, si no que se pueda ver,
paso a paso, la evolución de esta misma. 

Esto conlleva los siguiente:
\begin {itemize}
\item Necesidad de conocer en que paso de la simulación nos encontramos.
\item Llevar un historial de la inundación.
\item Realizar la inundación ordenadamente.
\end {itemize}

\subsection*{Visor Simple}
Para las primeras fases del desarrollo y para poder ver de una forma simple y
sin ayuda de programas externos, necesitamos un visor nativo del sistema, con
el que poder seguir el progreso y comportamiento del agua y de los demás
agentes implicados en la simulación.
\subsection*{Kml}

Una vez terminada la simulación, queremos poder tener un historial de todo lo
que haya ocurrido y la facilidad de poder navegar dentro de la simulación para
poder recrear el estado de la simulación en un momento preciso.

Para ello dispondremos de un sistema que guarde la información de cada
instantánea de la simulación dentro de un fichero Kml.

Hemos elegido Kml debido a que es un estándar y a la versatilidad y potencia
gráfica que nos proporciona.

Utilizar Kml también nos obligará a tratar con problemas de geometría
computacional, y de topología. 

\subsection*{Estadísticas}
Una buena forma de comparar entre los distintos comportamientos implementados
para el agua o las personas o cualquier otro agente, son las estadísticas, así
que tendremos que crear un sistema que sea capaz de recoger estadísticas de tal
forma que nos puedan dar información importante sobre la simulación y así poder
discernir entre que métodos o estrategias son mejores con respecto a que
objetivos.