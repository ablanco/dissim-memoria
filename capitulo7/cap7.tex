% Conclusiones y trabajo futuro

\chapter*{Conclusiones} \label{cap7}
\addcontentsline{toc}{chapter}{Conclusiones}

\begin{flushright}
\begin{minipage}{7.85cm}
    {\em Blah blah blah.} \\  Alguien
\end{minipage}
\end{flushright}

\vspace*{5mm}

\section*{Conclusiones}
Este ha sido, a lo largo de toda nuestra carrera, el proyecto mas largo, mas
complejo y mas ambicioso al que nos hemos enfrentado.

Al inicio del desarrollo teníamos claro que queríamos obtener un producto
interesante, al que le dedicaríamos mucho esfuerzo, pero del que obtendríamos
grandes beneficios.

Poner en practica las técnicas y métodos de desarrollo estudiados en clase,
buscar los limites de nuestras capacidades y sobre todo aprender.

En este proyecto utilizamos tecnologías totalmente nuevas para nosotros, JADE,
KML, servicios webs.
Para ello nos hemos vistos obligados a aprender a manejarnos con soltura entre
tecnologías nuevas, o con tecnologías ya consolidadas 
Además hemos tenido que enfrentarnos a problemas de integración, diseño, y
desarrollo, a medida que nuestro proyecto ha ido creciendo.

En lineas generales estamos muy contentos tanto del resultado final, como del
desarrollo.
\subsection*{Sobre el Simulador}
Para poder llevar a cabo un simulador, primero tuvimos que investigar sobre lo
que queríamos simular, luego investigar como se hace un simulador, y
finalmente, implementarlo.

Para ello nos hemos servido de gran cantidad de bibliografía con la que nos
hemos documentado sobre que características tiene que tener un simulador, y en
que aspectos puede un simulador ayudar a prevenir catástrofes y ser útil para
gestionar catástrofes.

\subsubsection*{Simular grandes zonas}
%muchos recursos, gran cantidad de volumen de datos
Nuestra intención es poder simular desastres en grandes extensiones de terreno,
como lo son las grandes ciudades, esto genera una enorme cantidad de datos,
muchos mas de los que en un principio habíamos pensado, esto nos obligó a
pensar en un rediseño de la aplicación para que fuera fácilmente escalable.

Esto nos ha obligado a tener que pensar en grande, estructuras de datos
eficientes y muy sencillas de mantener y ampliar en el futuro. 

También hemos tenido que pensar en sistemas eficientes de almacenamiento y
obtención de información, como por ejemplo técnicas de optimización y
codificación de los datos para ahorrar espacio en
memoria y aún así seguir manteniendo toda la expresividad posible.

En resumen, nos ha ayudado en pensar en grande.
\subsubsection*{Implementación}
%diseñar las clases y tal ... puedo enrollarme un rato ...
Este proyecto ha ido evolucionando constantemente, y hemos puesto una gran
cantidad de esfuerzo y tiempo en la estructura del proyecto, para que fuese lo
suficientemente robusta y dinámica como para aguantar todos los cambios.

Por ello, hemos tenido que encontrar el equilibrio entre la evolución, y el
volver a empezar, realizando los cambios mas pequeños posibles y añadiendo las
nuevas funcionalidades aprovechando los recursos ya creados y modificando las
pocas cosas que realmente necesitamos.

De esta experiencia hemos aprendido lo importante que es tener un buen diseño
inicial y que realmente merece la pena en invertir tiempo en pensarlos y
diseñarlo bien y lo importante que es controlar los cambios en tu código y el
impacto que tienen en el conjunto.

También nos hemos seguido ciertas normas de estilo, para que 
el código escrito por un desarrollador no le resultara totalmente desconocido
al otro. Para ayudar a que esto no pasara, también hemos aplicado técnicas
dinámicas de programación, como corregir el código escrito por otra persona o
realizar cierto seguimiento del código generado por el otro desarrollador.

Además nos ha resultado útil tener ciertas clases de test, con las que
probábamos las nuevas funcionalidades del código y a la vez tener una forma
fácil de averiguar el funcionamiento de la nueva funcionalidad.
\subsubsection*{Investigación}
%hablar del mendeley, de la importancia de la bibliografia ...
Este proyecto empezó siendo un simulador de movimientos de refugiados ante
amenazas de grupos armados.

Para poder llevar a cabo esta simulación, tuvimos a buscar entre
montañas de información, algo relativo a movimientos humanos relacionados con
campos de refugiados.

Pero también necesitábamos más cosas, como información sobre las necesidades
alimentarias, el desplazamiento medio de un grupo de refugiados y tener en
cuenta factores meteorológicos, geográficos, políticos y de conflictos cercanos
que pudieran modificar las trayectorias.

Por ello, más que un problema de simulación en sí, nos vimos con un problema de
falta de documentación, puesto que aunque la documentación relativa a este tipo
de desastre humanitario había crecido considerablemente en los últimos meses,
no se llegaba al nivel de detalle  suficiente.

Las posición de los campamentos, las fechas, las rutas recorridas y los recursos
que poseían los refugiados, esta información estaba ausente y simplemente
se limitaba a simples estadísticas de
supervivientes y enfermos que no servían de mucho para poder llevar a cabo una
simulación real del desarrollo.

Así que nos vimos obligados a abandonar el proyecto y redirigirlo a otras
catástrofes mas documentadas, como por ejemplo inundaciones en países
desarrollados, mas concretamente, por el volumen de datos y la precisión de
estos, la inundación del Huracán Katrina en nueva orleans\footnote{Huracán
Katrina : \url{http://es.wikipedia.org/wiki/Hurac\%C3\%A1n_Katrina}}.

Y de esta experiencia aprendimos a investigar, a buscar en las bases de datos
publicas de los gobiernos, de la ONU\footnote{Naciones Unidas :
\url{http://www.un.org/es/}} y a buscar entre montañas de información
los datos realmente importantes.
\subsection*{Personales}
Al pasar tanto tiempo desarrollando nuestra idea, e invertir tanto esfuerzo y
esperanzas en un sistema en el que ves el resultado de toda una carrera de
estudios, hemos aprendido el valor humano de lleva detrás un desarrollo
software.
\subsubsection*{Trabajo en grupo}
El trabajo con otras personas es imprescindible si quieres llevar a cabo un
desarrollo medianamente grande. Prácticamente para todo lo que quieras hacer,
si quieres que funcione y se utilice, tiene que pensar en que mas de una
persona pueda trabajar con el.

Para ello es muy importante la documentación, los comentarios en el código y
mantener buenas relaciones entre los miembros del grupo.

Establecer objetivos a corto plazo, llevar un control equitativo del reparto de
tareas y sobre todo compartir las opiniones y pedir opinión a los demás
desarrolladores sobre los problemas que surgen, buscar el consenso y
unificar el desarrollo son cosas muy importantes que hemos podido experimentar
en este desarrollo.

Simplemente, dos ojos ven mejor que uno solo, y si los dos miran al mismo
sitio, y hablan entre ellos, pues entonces seremos capaces de ver realmente
bien.
\subsubsection*{Comunidad de desarrollo libre}
Aunque para nuestro proyecto no podíamos aceptar colaboraciones externas,
creemos que el futuro de nuestra aplicación tiene mucho mas potencia en una
comunidad de desarrollo libre que en un cd entre una pila de papeles olvidados
en algún despacho.

Por ello apostamos por la comunidad.
\subsubsection*{Licencias Software}
La experiencia nos dice, que las licencias privativas no nos valían para
nuestro propósito, y utilizar una licencia libre como \hyperref[ap1]{\bf
GPL} nos era mucho mas
favorable en el ámbito de la investigación y el desarrollo.

Puesto que nuestra aplicación no esta enfocada a ser vendida, pero si a ser
utilizada por todo el mundo.

Dejando a un lado nuestras opiniones personales, creemos que este tipo de
licencia es la que mas se ajustaba a nuestras necesidades y expectativas.
\subsection*{Ampliaciones y mejoras}
Por  mucho que nos hayamos implicado en este proyecto, somos perfectamente
conscientes de que tenemos que terminarlo en algún momento, y lejos de
pretender entregar el producto perfecto o que simplemente compile, le hemos
dado prioridad a llegar a un nivel de calidad y acabado óptimo en relación al
tiempo disponible.

Y aunque estamos muy satisfechos con el resultado final, siempre se puede
mejorar.
\subsubsection*{Rendimiento}
El rendimiento de algunas funciones se podría mejorar, algunas búsquedas no son
del todo óptimas y en algunos cálculos se ha buscado más la claridad y la
simpleza que la optimización y la rapidez. Sin embargo no forman parte del
núcleo crítico de la aplicación, por lo que hemos decidido dejarlos a una
posible ampliación. 

Estos métodos aparecen indicados y comentados en el código para una futura
reimplementación mas eficiente.
\subsubsection*{Otras catástrofes}
Como ya hemos dicho antes, nuestro proyecto no se limita solo a las
inundaciones, y fácilmente podría ser ampliado a simular otro tipo de
catástrofes, como terremotos, incendios, sequías ...

La estructura de datos ha sido creada para que añadir nuevas catástrofes sea
simple, permitiendo incluso simular catástrofes múltiples.
\subsubsection*{Interfaz Gráfica}
Somos perfectamente conscientes de que si queremos que nuestra aplicación sea
utilizada por terceras personas ajenas al mundo universitario y de la
investigación la implementación de una interfaz gráfica es esencial para
facilitar el manejo de esta aplicación.

Sin embargo en este desarrollo no lo hemos considerado prioritario ya que
el script utilizado para lanzar la aplicación es bastante simple y flexible.

%%% Local Variables:
%%% mode: latex
%%% TeX-master: "../dissim"
%%% End: