\chapter*{Conclusiones y Trabajo futuro} \label{cap7}
\addcontentsline{toc}{chapter}{Conclusiones y Trabajo futuro}

\begin{flushright}
\begin{minipage}{7.85cm}
    {\em Es débil porque no ha dudado bastante y ha querido llegar a
    conclusiones.} \\ Miguel de Unamuno
\end{minipage}
\end{flushright}

\vspace*{5mm}

\section*{Conclusiones}

Este ha sido, a lo largo de toda nuestra carrera, el proyecto más largo, más
complejo y más ambicioso al que nos hemos enfrentado.

Al inicio del desarrollo teníamos claro que queríamos obtener un producto
interesante, al que le dedicaríamos mucho esfuerzo, pero del que obtendríamos
grandes beneficios. Poner en practica las técnicas y métodos de desarrollo
estudiados en clase, buscar los límites de nuestras capacidades y, sobre todo,
aprender.

En este proyecto utilizamos tecnologías totalmente nuevas para nosotros: JADE,
KML, \LaTeX{}, etc. Por ello nos hemos vistos obligados a aprender a manejarnos
con soltura entre tecnologías nuevas y tecnologías ya consolidadas. Además hemos
tenido que enfrentarnos a problemas de integración, diseño, y desarrollo, a
medida que nuestro proyecto ha ido creciendo.

En líneas generales estamos muy contentos tanto del resultado final como del
proceso de desarrollo.

\subsection*{Sobre el Simulador}

Para poder construir un simulador primero tuvimos que investigar sobre lo
que queríamos simular, luego investigar cómo se hace un simulador, y
finalmente, implementarlo.

Para ello nos hemos servido de gran cantidad de bibliografía con la que nos
hemos documentado sobre que características tiene que tener un simulador, y en
que aspectos puede un simulador ayudar a prevenir catástrofes y ser útil para
gestionarlas.

\subsubsection*{Simular Grandes Zonas}

Nuestra intención era poder simular desastres en grandes extensiones de terreno,
como lo son las grandes ciudades. Esto genera una enorme cantidad de datos,
muchos más de los que en un principio habíamos pensado. Esto nos obligó a
pensar en un rediseño de la aplicación para que fuera fácilmente escalable.

Hemos tenido, pues, que pensar en grande para diseñar estructuras de datos
eficientes y muy sencillas de mantener y ampliar en el futuro. De la misma
manera hemos pensado sistemas eficientes de almacenamiento y obtención de
información, como por ejemplo técnicas de optimización y codificación de los
datos para ahorrar espacio en memoria, y aún así seguir manteniendo toda la
expresividad posible.

En resumen, el proyecto nos ha ayudado a aprender a tener una visión global del
diseño y manera de desarrollar los grandes proyectos.

\subsubsection*{Implementación}

Este proyecto ha ido evolucionando constantemente, y por ello hemos puesto una
gran cantidad de esfuerzo y tiempo en la estructura del mismo, para que fuese lo
suficientemente robusta y dinámica como para aguantar todos los cambios.

Hemos tenido que encontrar el equilibrio entre la evolución, y el ``volver a
empezar'', realizando los cambios más pequeños posibles y añadiendo las
nuevas funcionalidades aprovechando los recursos ya creados, modificando las
pocas cosas que realmente era necesario cambiar.

De esta experiencia hemos aprendido lo importante que es tener un buen diseño
inicial, y que realmente merece la pena invertir tiempo en hacerlo bien.
También hemos comprendido lo importante que es controlar los cambios en tu
código, y el impacto que tienen en el conjunto.

Hemos seguido ciertas normas de estilo, para que el código escrito por un
desarrollador no le resultara totalmente desconocido al otro. Para ayudar a que
esto no pasara también hemos aplicado técnicas dinámicas de programación, como
corregir el código escrito por otra persona o realizar cierto seguimiento del
código generado por el otro desarrollador.

Además nos ha resultado útil tener ciertas clases de test, con las que
probábamos las nuevas funcionalidades del código y, a la vez, tener una forma
fácil de averiguar el funcionamiento de la nueva funcionalidad.

\subsubsection*{Investigación}

% TODO Esto no son conclusiones sobre investigación, mover a otro capítulo

Este proyecto empezó siendo un simulador de movimientos de refugiados ante
amenazas de grupos armados. Para poder llevar a cabo este tipo de simulación,
era necesario buscar, entre montañas de información, algo relativo a movimientos
humanos relacionados con campos de refugiados.

Pero también necesitábamos más cosas, como información sobre las necesidades
alimentarias o el desplazamiento medio de un grupo de refugiados, además de
tener en cuenta factores meteorológicos, geográficos, políticos y conflictos
cercanos que pudieran modificar las rutas.

Por ello, más que un problema de simulación en sí, nos encontramos con un
problema de falta de documentación. Aunque la documentación relativa a este tipo
de desastre humanitario había crecido considerablemente en los últimos años,
no se llegaba al nivel de detalle suficiente, o al menos la información pública
y accesible no tenía dicho nivel.

Así que nos vimos obligados a abandonar dicha idea, y redirigir el proyecto a
otro tipo de catástrofes más documentadas, como por ejemplo inundaciones en
países desarrollados. En concreto, por el volumen de datos y la precisión de
estos, la inundación posterior al paso del Huracán Katrina en Nueva Orleans.

% TODO dejar este párrafo (el siguiente)

Y de esta experiencia aprendimos a investigar, a buscar en las bases de datos
publicas de los gobiernos, de la ONU, y a buscar entre montañas de información
los datos realmente importantes.

\subsection*{Personales}

Al pasar tanto tiempo desarrollando nuestra idea, e invertir tanto esfuerzo y
esperanzas en un sistema en el que ves el resultado de toda una carrera de
estudios, hemos aprendido el valor humano de lleva detrás un desarrollo
software.

\subsubsection*{Trabajo en grupo}

El trabajo con otras personas es imprescindible si quieres llevar a cabo un
desarrollo medianamente grande. Prácticamente para todo lo que quieras hacer,
si quieres que funcione y se utilice, tiene que pensar en que mas de una
persona pueda trabajar con el.

Para ello es muy importante la documentación, los comentarios en el código y
mantener buenas relaciones entre los miembros del grupo.

Establecer objetivos a corto plazo, llevar un control equitativo del reparto de
tareas y sobre todo compartir las opiniones y pedir opinión a los demás
desarrolladores sobre los problemas que surgen, buscar el consenso y
unificar el desarrollo son cosas muy importantes que hemos podido experimentar
en este desarrollo.

Simplemente, dos ojos ven mejor que uno solo, y si los dos miran al mismo
sitio, y hablan entre ellos, pues entonces seremos capaces de ver realmente
bien.
\subsubsection*{Comunidad de desarrollo libre}
Aunque para nuestro proyecto no podíamos aceptar colaboraciones externas,
creemos que el futuro de nuestra aplicación tiene mucho mas potencia en una
comunidad de desarrollo libre que en un cd entre una pila de papeles olvidados
en algún despacho.

Por ello apostamos por la comunidad.
\subsubsection*{Licencias Software}
La experiencia nos dice, que las licencias privativas no nos valían para
nuestro propósito, y utilizar una licencia libre como \hyperref[ap1]{\bf
GPL} nos era mucho mas
favorable en el ámbito de la investigación y el desarrollo.

Puesto que nuestra aplicación no esta enfocada a ser vendida, pero si a ser
utilizada por todo el mundo.

Dejando a un lado nuestras opiniones personales, creemos que este tipo de
licencia es la que mas se ajustaba a nuestras necesidades y expectativas.
\section*{Ampliaciones y mejoras}
Por  mucho que nos hayamos implicado en este proyecto, somos perfectamente
conscientes de que tenemos que terminarlo en algún momento, y lejos de
pretender entregar el producto perfecto o que simplemente compile, le hemos
dado prioridad a llegar a un nivel de calidad y acabado óptimo en relación al
tiempo disponible.

Y aunque estamos muy satisfechos con el resultado final, siempre se puede
mejorar.
\subsection*{Rendimiento}
El rendimiento de algunas funciones se podría mejorar, algunas búsquedas no son
del todo óptimas y en algunos cálculos se ha buscado más la claridad y la
simpleza que la optimización y la rapidez. Sin embargo no forman parte del
núcleo crítico de la aplicación, por lo que hemos decidido dejarlos a una
posible ampliación. 

Estos métodos aparecen indicados y comentados en el código para una futura
reimplementación mas eficiente.
\subsection*{Otras catástrofes}
Como ya hemos dicho antes, nuestro proyecto no se limita solo a las
inundaciones, y fácilmente podría ser ampliado a simular otro tipo de
catástrofes, como terremotos, incendios, sequías ...

La estructura de datos ha sido creada para que añadir nuevas catástrofes sea
simple, permitiendo incluso simular catástrofes múltiples.
\subsection*{Interfaz Gráfica}
Somos perfectamente conscientes de que si queremos que nuestra aplicación sea
utilizada por terceras personas ajenas al mundo universitario y de la
investigación la implementación de una interfaz gráfica es esencial para
facilitar el manejo de esta aplicación.

Sin embargo en este desarrollo no lo hemos considerado prioritario ya que
el script utilizado para lanzar la aplicación es bastante simple y flexible.

%%% Local Variables:
%%% mode: latex
%%% TeX-master: "../dissim"
%%% End: