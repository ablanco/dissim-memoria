\chapter{Conclusiones y Trabajo Futuro} \label{cap7}
%\addcontentsline{toc}{chapter}{Conclusiones y Trabajo Futuro}

\begin{flushright}
\begin{minipage}{7.85cm}
    {\em Es débil porque no ha dudado bastante y ha querido llegar a
    conclusiones.} \\ Miguel de Unamuno
\end{minipage}
\end{flushright}

\vspace*{5mm}

\section{Conclusiones}

Este ha sido, a lo largo de toda nuestra carrera, el proyecto más largo, más
complejo y más ambicioso al que nos hemos enfrentado.

Al inicio del desarrollo teníamos claro que queríamos obtener un producto
interesante, al que le dedicaríamos mucho esfuerzo, pero del que obtendríamos
grandes beneficios. Poner en practica las técnicas y métodos de desarrollo
estudiados en clase, buscar los límites de nuestras capacidades y, sobre todo,
aprender.

En este proyecto utilizamos tecnologías totalmente nuevas para nosotros: JADE,
KML, \LaTeX{}, etc. Por ello nos hemos vistos obligados a aprender a manejarnos
con soltura entre tecnologías nuevas y tecnologías ya consolidadas. Además hemos
tenido que enfrentarnos a problemas de integración, diseño, y desarrollo, a
medida que nuestro proyecto ha ido creciendo.

En líneas generales estamos muy contentos tanto del resultado final como del
proceso de desarrollo.

\subsection{Sobre el Simulador}

Para poder construir un simulador primero tuvimos que investigar sobre lo
que queríamos simular, luego investigar cómo se hace un simulador, y
finalmente, implementarlo.

Para ello nos hemos servido de gran cantidad de bibliografía con la que nos
hemos documentado sobre que características tiene que tener un simulador, y en
que aspectos puede un simulador ayudar a prevenir catástrofes y ser útil para
gestionarlas.

\subsubsection{Simular Grandes Zonas}

Nuestra intención era poder simular desastres en grandes extensiones de terreno,
como lo son las grandes ciudades. Esto genera una enorme cantidad de datos,
muchos más de los que en un principio habíamos pensado. Esto nos obligó a
pensar en un rediseño de la aplicación para que fuera fácilmente escalable.

Hemos tenido, pues, que pensar en grande para diseñar estructuras de datos
eficientes y muy sencillas de mantener y ampliar en el futuro. De la misma
manera hemos pensado sistemas eficientes de almacenamiento y obtención de
información, como por ejemplo técnicas de optimización y codificación de los
datos para ahorrar espacio en memoria, y aún así seguir manteniendo toda la
expresividad posible.

En resumen, el proyecto nos ha ayudado a aprender a tener una visión global del
diseño y manera de desarrollar los grandes proyectos.

\subsubsection{Implementación}

Este proyecto ha ido evolucionando constantemente, y por ello hemos puesto una
gran cantidad de esfuerzo y tiempo en la estructura del mismo, para que fuese lo
suficientemente robusta y dinámica como para aguantar todos los cambios.

Hemos tenido que encontrar el equilibrio entre la evolución, y el ``volver a
empezar'', realizando los cambios más pequeños posibles y añadiendo las
nuevas funcionalidades aprovechando los recursos ya creados, modificando las
pocas cosas que realmente era necesario cambiar.

De esta experiencia hemos aprendido lo importante que es tener un buen diseño
inicial, y que realmente merece la pena invertir tiempo en hacerlo bien.
También hemos comprendido lo importante que es controlar los cambios en tu
código, y el impacto que tienen en el conjunto.

Hemos seguido ciertas normas de estilo, para que el código escrito por un
desarrollador no le resultara totalmente desconocido al otro. Para ayudar a que
esto no pasara también hemos aplicado técnicas dinámicas de programación, como
corregir el código escrito por otra persona o realizar cierto seguimiento del
código generado por el otro desarrollador.

Además nos ha resultado útil tener ciertas clases de test, con las que
probábamos las nuevas funcionalidades del código y, a la vez, tener una forma
fácil de averiguar el funcionamiento de la nueva funcionalidad.

\subsubsection{Investigación}

\hyperref[inicios]{Al principio del proyecto}, cuando aún no estaba ni siquiera
del todo definido, comenzamos una labor de investigación. Dedicamos semanas a
la búsqueda y análisis de documentación.

Afortunadamente hoy en día, gracias a la informática y a internet, uno dispone
de cientos de fuentes de documentos. Por otra parte esto provoca que sea
difícil organizar y mantener tanta documentación, y que haya que saber
seleccionar qué documentos son los que interesan.

Para el primer problema nos apoyamos en la herramienta
Mendeley\footnote{\url{http://www.mendeley.com/}}, que nos ayudó a trabajar con
la bibliografía. Pero para el segundo tuvimos que aplicar nuestras propias
habilidades de síntesis y análisis.

Y de esta experiencia aprendimos a investigar, a buscar en las bases de datos
publicas de los gobiernos, de la ONU, y a buscar entre montañas de información
los datos realmente importantes.

\subsection{Personales}

Al pasar tanto tiempo desarrollando nuestra idea, e invertir tanto esfuerzo y
esperanzas en un sistema en el que ves el resultado de toda una carrera de
estudios, hemos aprendido el valor humano que lleva detrás un desarrollo
software.

\subsubsection{Trabajo en Grupo}

El trabajo con otras personas es imprescindible si quieres llevar a cabo un
desarrollo medianamente grande. Para ello es muy importante la documentación,
los comentarios en el código y mantener buenas relaciones entre los miembros del
grupo.

Establecer objetivos a corto plazo, llevar un control equitativo del reparto de
tareas y, sobre todo, compartir las opiniones y pedir consejo a los demás
desarrolladores sobre los problemas que surgen. Buscar el consenso y unificar el
desarrollo son aspectos muy importantes que hemos podido experimentar en este
proyecto.

\subsubsection{Comunidad de Desarrollo Libre}

Aunque para nuestro proyecto no podíamos aceptar colaboraciones externas,
creemos que el futuro de nuestra aplicación tiene mucho más potencial en una
comunidad de desarrollo libre que en un CD olvidado entre una pila de papeles
en algún despacho.

Por ello apostamos por la comunidad.

\subsubsection{Licencias Software}

La experiencia nos dice que las licencias privativas no nos valían para
nuestro propósito, y utilizar una licencia libre como la \hyperref[ap1]{\bf GPL}
nos era mucho más favorable, sobre todo en el ámbito de la investigación y el
desarrollo.

Dejando a un lado nuestras opiniones personales, creemos que este tipo de
licencia es la que más se ajustaba a nuestras necesidades y expectativas.

\section{Ampliaciones y Mejoras}

Por mucho que nos hayamos implicado en este proyecto, somos perfectamente
conscientes de que tenemos que terminarlo en algún momento. Lejos de querer
pretender entregar el producto perfecto, o que simplemente compile, le hemos
dado prioridad a llegar a un nivel de calidad y acabado óptimo en relación al
tiempo disponible.

Y aunque estamos muy satisfechos con el resultado final, siempre se puede
mejorar y ampliar.

\subsection{Rendimiento}

El rendimiento de algunas funciones se podría mejorar, algunas búsquedas no son
del todo óptimas y en algunos cálculos se ha buscado más la claridad y la
simpleza que la optimización y la rapidez. Sin embargo, no forman parte del
núcleo crítico de la aplicación, por lo que hemos decidido dejarlos para una
posible ampliación. 

Estos métodos aparecen indicados y comentados en el código para una futura
implementación más eficiente.

\subsubsection{Memoria}

El principal problema que hemos encontrado a la hora de simular es el consumo
de memoria RAM, que se dispara enormemente para escenarios de gran tamaño.
Aunque el simulador funciona muy bien para escenarios reducidos, consideramos
una prioridad mejorar el consumo y manejo de la memoria, de manera que sea más
fácilmente escalable y se puedan simular áreas tan grandes como se desee.

\subsection{Otras catástrofes}

Como ya hemos dicho antes, nuestro proyecto no se limita sólo a las
inundaciones, se trata de una plataforma para poder simular múltiples tipos de
desastres. Puede ser ampliado fácilmente para simular otro tipo de catástrofe
como puedan ser terremotos, incendios, sequías, etc.

La estructura de datos ha sido creada para que añadir nuevas catástrofes sea
simple. Para la implementación de un nuevo desastre es necesario extender
ciertas clases, y realizar modificaciones menores a otras.

Hay que extender las clases {\em util.Scenario} y {\em util.HexagonalGrid}. En
ambos casos hay que sobreescribir los métodos que se consideren oportunos y
añadir nuevos según las necesidades del nuevo desastre. Habría, probablemente,
que crear agentes nuevos específicos de la catástrofe, pero muchos se pueden
reutilizar, como los agentes {\bf Creador}, {\bf Entorno}, {\bf Actualización},
{\bf Reloj} o {\bf Peatones}.

Para los peatones habría que crear comportamientos nuevos adecuados para el
desastre, extendiendo la clase {\em behaviours.people.PedestrianBehav}. Sería
necesario añadir al agente {\bf Creador} una sección donde crease los agentes
específicos, y posiblemente hubiera que añadir comportamientos específicos a
los agentes {\bf Entorno}.

En cuanto a la visualización de resultados, habría que adaptar el generador de
KML para que mostrase las particularidades de la catástrofe. Lo mismo ocurre
con el visor bidimensional.

\subsection{Interfaz Gráfica}

Comprendemos que si queremos que nuestra aplicación sea utilizada por terceras
personas ajenas al mundo universitario y de la investigación, la implementación
de una interfaz gráfica es esencial para facilitar el manejo.

Sin embargo en este desarrollo no lo hemos considerado prioritario, ya que
el script utilizado para lanzar la aplicación es bastante simple y flexible.

%%% Local Variables:
%%% mode: latex
%%% TeX-master: "../dissim"
%%% End: