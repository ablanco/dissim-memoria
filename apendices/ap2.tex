\chapter{Concurso Universitario de Software Libre} \label{ap2}
\addcontentsline{toc}{chapter}{Concurso Universitario de Software Libre}

\vspace*{5mm}

\section{Introducción}

En este apéndice comentamos y exponemos los datos y bases del Concurso
Universitario de Software Libre (CUSL), en concreto en lo referente a su cuarta
edición, que es en la que inscribimos el proyecto.

Toda esta información ha sido extraída de la web del concurso:
\url{http://www.concursosoftwarelibre.org/}

\subsection{¿Qué es...?}

Es un concurso de desarrollo de software, hardware y documentación técnica
libre, en el que pueden participar estudiantes universitarios de primer,
segundo y tercer ciclo; así como estudiantes no universitarios de bachillerato,
grado medio y superior del ámbito estatal español.

Para liberar los proyectos se emplea principalmente la forja de
RedIRIS\footnote{Forja IRIS-Libre: \url{https://forja.rediris.es/}}. Los
participantes además disponen de un blog donde contarán su experiencia en el
desarrollo durante el curso académico.

\subsection{¿Qué Objetivos Persigue?}

El objetivo principal del Concurso Universitario de Software Libre es fomentar
la creación, y contribuir a la consolidación, de la comunidad del Software Libre
en la Universidad.

\subsection{Premios Locales}

Además del Premio General a nivel estatal, desde la segunda edición del
concurso, existen los Premios Locales.

La Universidad e instituciones vinculadas al mundo del software libre están
invitadas a formar parte de la comunidad del Concurso Universitario de Software
Libre.

Para ello disponen de la posibilidad de premiar a los mejores proyectos
participantes de su ámbito, así como de realizar una fase final local mediante
la modalidad de Premio Local.

De esta forma, sus participantes obtendrán un mayor reconocimiento, se dará
máxima visibilidad a las universidades participantes en la comunidad.

\section{Descripción General}

\begin{enumerate}
\item El Concurso Universitario de Software Libre es un concurso de desarrollo
de software, hardware y documentación técnica libre que consiste en la
elaboración y presentación de un proyecto de Software Libre desarrollado
íntegramente con una implementación libre de cualquier lenguaje de programación.
\item El desarrollo del CUSL consta de las siguientes fases:
    \begin{enumerate}
    \item Fase de inscripción.
    \item Fase de desarrollo.
    \item Fase final.
    \end{enumerate}
\item El máximo de participantes será de tres personas por proyecto.
\item Sólo se podrá participar en un proyecto a la vez.
\item Los proyectos debe estar íntegramente desarrollados con la implementación
en Software Libre de cualquier lenguaje de programación y la licencia elegida
deberá estar comprendida dentro del conjunto de licencias consideradas de
Software Libre por el proyecto GNU (lista en castellano no actualizada). Las
librerías utilizadas también deberán estar bajo este conjunto de licencias.
\item La organización pondrá a disposición de los participantes una forja en la
que se alojarán los proyecto participantes en el CUSL. Los participantes también
dispondrán de un blog, en el que quedará reflejado la evolución del proyecto,
las dificultades encontradas y las decisiones tomadas durante la fase de
desarrollo. Tanto la forja como el blog serán empleados para evaluar la
evolución de los proyectos.
\item El comité de evaluación valorará los siguientes aspectos:
    \begin{enumerate}
    \item {\bf Impacto del proyecto} en la comunidad del software libre y la
    sociedad (30\% de la puntuación). Se valorará positivamente que el proyecto
    genere un tejido social (comunidad) con un interés común en hacer que dicho
    proyecto avance. Esto comprende aspectos como la facilidad para realizar
    trabajos relacionados y/o derivados, su impacto en la esfera socio-política
    y las estrategias de difusión del proyecto en la comunidad del software
    libre.
    \item {\bf Calidad del desarrollo y uso de la forja} (30\% de la
    puntuación). Se valorará positivamente una buena planificación, así como la
    realización de un buen diseño e implementación del proyecto. Una buena
    estrategia en el lanzamiento de versiones será igualmente bien considerada.
    Además, un buen uso de las herramientas disponibles en la forja, tales como
    las listas de correo y el sistema de gestión de código (subversion), serán
    igualmente valorados positivamente.
    \item {\bf Documentación del proyecto} (25\% de la puntuación): La
    documentación es un aspecto importante. Se valorará positivamente la
    realización de documentación de usuario final (detallando la instalación y
    configuración) y de desarrollo (ofreciendo la información pertinente para
    facilitar la realización de trabajos derivados y adaptaciones).
    Grado de finalización (15\% de la puntuación): Se valorará positivamente,
    aunque con menor valor que otros puntos anteriormente descritos, el grado de
    finalización y usabilidad del proyecto.
    \end{enumerate}
\end{enumerate}

\section{Participación}

Nuestro proyecto ha participado en la IV edición del Concurso Universitario de
Software Libre, y nos ha sido otorgada una Mención Especial en el III Premio
Local de Sevilla.